\documentclass{tutorial}

\subject{Quantum Field Theory}
\subtitle{Problem Sheet 4}
\title{Scalar Correlators}
\author{L.~Del Debbio}
\date{October 2018}

\begin{document}
\maketitle

\begin{enumerate}
  \item \emph{ Feynman rules - 1}\\

    Use the Feynman rules in momentum space to compute
    $G^{(2,2)}_b$. Check that you get the same result by performing a
    Fourier transform of the result in position space.

    \bigskip

  \item \emph{ Scattering amplitude} \\
    
      Compute the amplitude for the scattering process
      \[
        p_1 p_2 \longrightarrow p_1' p_2'
      \]
      at order $g^2$ in the $\phi^3$ scalar theory. 


    \bigskip
    
  \item \emph{ LSZ reduction for 2 to 2 processes}\\
    
    A particle localised in momentum space near $\mathbf{k}_1$ is
    created in $D=4$ dimensions by the operator
    \begin{align}
      \label{eq:1}
      a_1^\dagger &= \int \dd[3]{k}  f_1(\mathbf{k}) a^\dagger(\mathbf{k})\, ,
    \end{align}
    where $f$ is some function peaked at $\mathbf{k}_1$, and
    $a^\dagger(\mathbf{k})$ is the creation operator in the free
    theory. In the interacting theory, we shall assume that a
    time-dependent creation operator is defined as
    \begin{align}
      \label{eq:2}
      a^\dagger(\mathbf{k},t) = -i \int \dd[3]{x}\, e^{-ik\cdot x}
      \overleftrightarrow{\partial_0} \phi(x)\, .
    \end{align}
    Show that
    \begin{align}
      \label{eq:3}
      a_1^\dagger(+\infty) - a_1^\dagger(-\infty) = 
      -i \int \dd[3]{k}\, f_1(\mathbf{k}) \int d^4x\,
      e^{-ik\cdot x} (\partial^2+m^2) \phi(x).
    \end{align}
 
    The scattering amplitude for a $2\rightarrow 2$ process can be
    written as: 
    \begin{align}
      \label{eq:4}
      \langle k_1' k_2';\mathrm{out} | k_1 k_2; \mathrm{in} \rangle 
      &= \langle 0 | T\left(
        a_{1'}(+\infty) a_{2'}(+\infty) 
        a_1^\dagger(-\infty) a_2^\dagger(-\infty)
        \right) |0 \rangle\, .
    \end{align}
    Show that
    \begin{align}
      \label{eq:5}
      \langle k_1' k_2';\mathrm{out} | k_1 k_2; \mathrm{in} \rangle 
      =&  i^{2+2} 
         \int d^4x_1\, e^{-i k_1\cdot x_1} \left(\partial_1^2 + m^2\right)  
         \int d^4x_2\, e^{-i k_2\cdot x_2} \left(\partial_2^2 +
         m^2\right)  \nonumber \\
      & \times \int d^4x_1'\, e^{i k_1'\cdot x_1'} \left(\partial_{1'}^2 + m^2\right)  
         \int d^4x_2'\, e^{i k_2'\cdot x_2'} \left(\partial_{2'}^2 +
        m^2\right) \nonumber \\
      & \times \expval{T\left(
        \phi(x_1) \phi(x_2) \phi(x_1') \phi(x_2')
        \right) }{0}.        
    \end{align}

\item \emph{ Generalised LSZ}

  Generalise the LSZ reduction for arbitrary numbers of particles in
  the initial and final state. 

\end{enumerate}
\end{document}
