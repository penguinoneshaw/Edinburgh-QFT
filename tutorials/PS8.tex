\documentclass{tutorial}

\subject{Quantum Field Theory}
\subtitle{Problem Sheet 8}
\title{Renormalization}
\author{L.~Del Debbio}
\date{November 2018}
% 
\usepackage[UKenglish]{babel}
\usepackage[UKenglish]{isodate}
\addto\captionsUKenglish{\renewcommand{\chaptername}{Lecture}}

\pagestyle{headings}

\usepackage{mathpazo}

\usepackage{amsmath}
\usepackage{amsfonts}
\usepackage{amssymb}
\usepackage{dsfont}
\usepackage{pifont}
%\usepackage{bbold}
\usepackage{graphicx}
\usepackage{epstopdf}
\usepackage{epsfig}
%\usepackage{bibunits}
%\usepackage{theorem}
\usepackage[framed]{ntheorem}
\usepackage{framed}
%\usepackage{showlabels}
\usepackage{makeidx}
\usepackage{simplewick}
\usepackage{tikz-feynman}
\usepackage{slashed}
\usepackage{appendix}
\usepackage[hidelinks]{hyperref}


\tikzfeynmanset{compat=1.0.0}  

\newcommand{\tick}{\ding{52}}
\newcommand{\notick}{\ding{56}}
\newcommand{\D}{\displaystyle}
\renewcommand{\chaptername}{Lecture}

\def\bfx{{\mathbf x}}
\def\bfxp{{\mathbf x^\prime}}
\def\bfy{{\mathbf y}}
\def\bfyp{{\mathbf y^\prime}}
\def\bfp{{\mathbf p}}
\def\bfpp{{\mathbf p^\prime}}
\def\ddt{\frac{d}{dt}}
\def\ddtt{\frac{d^2}{dt^2}}
\def\ie{\emph{ i.e.}\ }
\def\eg{\emph{ e.g.}\ }
\def\viz{\emph{ viz.}\ }
\def\matF{\mathcal F}
\def\matE{\mathcal E}
\def\GL{\mathrm{GL}}
\def\kpsi{|\psi\rangle}
\def\kpsione{|\psi_1\rangle}
\def\kpsitwo{|\psi_2\rangle}
\def\kpsionep{|\psi_1^\prime\rangle}
\def\kpsitwop{|\psi_2^\prime\rangle}
\def\kpsii{|\psi_i\rangle}
\def\kpsin{|\psi_n\rangle}
\def\kpsip{|\psi^\prime\rangle}
\def\bpsi{\langle\psi |}
\def\bpsione{\langle\psi_1 |}
\def\bpsitwo{\langle\psi_2 |}
\def\bpsii{\langle\psi_i |}
\def\bpsip{\langle\psi^\prime |}
\def\kphi{|\phi\rangle}
\def\kphione{|\phi_1\rangle}
\def\kphitwo{|\phi_2\rangle}
\def\kphii{|\phi_i\rangle}
\def\kphip{|\phi^\prime\rangle}
\def\bphi{\langle\phi |}
\def\bphione{\langle\phi_1 |}
\def\bphitwo{\langle\phi_2 |}
\def\bphii{\langle\phi_i |}
\def\bphip{\langle\phi^\prime |}
\def\bchi{\langle\chi |}
\def\bchione{\langle\chi_1 |}
\def\bchitwo{\langle\chi_2 |}
\def\bchii{\langle\chi_i |}
\def\bchip{\langle\chi^\prime |}
\def\kjm{|j,m\rangle}
\def\tr{\mathrm{Tr}}
\def\id{\mathds{1}}
{\theoremstyle{plain} \theorembodyfont{\rmfamily} \newframedtheorem{Ex}{Exercise}[section]}
{\theoremstyle{plain} \theorembodyfont{\rmfamily} \newtheorem{Def}{Definition}[section]}
{\theoremstyle{plain} \theorembodyfont{\rmfamily} \newtheorem{Thm}{Theorem}[section]}

\newcommand{\clearemptydoublepage}{\newpage{\pagestyle{empty}\cleardoublepage}}
\newcommand{\HRule}{\rule{\linewidth}{0.5mm}}
\newcommand{\iu}{\underline{i}}
\newcommand{\ju}{\underline{j}}
\newcommand{\ku}{\underline{k}}
\newcommand{\ru}{\underline{r}}
\newcommand{\pu}{\underline{p}}
\newcommand{\Lu}{\underline{L}}
\newcommand{\Ju}{\underline{J}}
\newcommand{\lap}{\nabla^2}
\newcommand{\ad}{\hat{a}}
\newcommand{\ac}{\hat{a}^\dagger}
\newcommand{\re}{\mathrm{Re}}
\newcommand{\ket}[1]{| #1 \rangle}
\newcommand{\bra}[1]{\langle #1 |}
\newcommand{\braket}[2]{\langle #1 | #2 \rangle}
\newcommand{\pref}[1]{(\ref{#1})}
\newcommand{\Eqref}[1]{Eq.~(\ref{#1})}
\newcommand{\del}{\v{\nabla}}				% Underlined del
\def\plusheight{-\the\dimexpr\fontdimen22\textfont2\relax}
\newcommand{\mphys}{m_\mathrm{phys}}
\newcommand{\psibar}{\bar{\psi}}
\newcommand{\etabar}{\bar{\eta}}
\newcommand{\munu}{{\mu\nu}}
\newcommand{\tDelta}{\tilde{\Delta}}
\newcommand{\SProp}[1]{\frac{1}{#1^2-m^2+i\epsilon}}
\newcommand{\ESProp}[2]{\frac{1}{\left(#1^2+m^2\right)^#2}}

\newcommand{\tphi}{\tilde{\phi}}
\newcommand{\tj}{\tilde{J}}
\newcommand{\tchi}{\tilde{\chi}}
\newcommand{\psibar}{\bar{\psi}}
\newcommand{\etabar}{\bar{\eta}}
\newcommand{\munu}{{\mu\nu}}

\begin{document}
\maketitle

\begin{enumerate}
\item \emph{ One-loop vertex}

  Write the one-loop contribution to the three-point function for the
  $\phi^3$ theory in $D$ dimensions using the Feynman rules for
  correlators. Check the computation of the diagram discussed in the
  lectures. 
  
\item \emph{ $\phi^4$ theory}

  The $\phi^4$ theory is defined by the interaction term
  \[
    \frac{\lambda}{4!} \int d^Dx\, \phi(x)^4\, .
  \]
  Compute the one-loop contribution to the scalar propagator at
  $\mathcal{O}(\lambda)$ in $D=4$. Identify the relevant diagram(s), compute the symmetry
  factor(s), and write them as integral(s) in momentum space. Regularise
  the integral(s) using DimReg, and discuss the divergences.

  Write the renormalization conditions for the field and the mass in
  the on-shell renormalization scheme, and determine the corresponding
  counter terms at $\mathcal{O}(\lambda)$.

\item \emph{ Superficial degree of divergence}

  Find the superficial degree of divergence of a diagram with $E$
  external legs in $D$ dimensions for a scalar $\phi^n$ theory, and for
  the Yukawa theory discussed in PS6.
  
\item \emph{ Four-point 1PI vertex}
  
  Check the computation of $V^{(4)}$ in $\phi^3$ in $D=6$.
  
\item \emph{ Renormalizable interactions in $D=4$}
  
  Consider fermionic interactions of the form $g_n (\psibar\psi)^n$,
  for $n \geq 2$. Find the mass dimension of $g_n$ in $D$ dimensions.
  
  Consider interactions of the form $g_{nm} \phi^m
  (\psibar\psi)^n$. Find the mass dimension of $g_{nm}$ in $D$
  dimensions.
  
  List the renormalizable interactions for $D=4$. Do they look
  familiar? 

\item \emph{ Two-particle amplitude at one-loop}

  Using the skeleton expansion compute the one-loop contribution to
  the scattering process:
  \[
    p_1 p_2 \rightarrow p_1' p_2'\, .
  \]

  The tree-level contribution was already computed in PS4, and the
  four-point vertex $V^{(4)}$ has been discussed in lectures.
  
\end{enumerate}
\end{document}
