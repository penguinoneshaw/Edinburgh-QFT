\documentclass{tutorial}

\subject{Quantum Field Theory}
\subtitle{Problem Sheet 5}
\title{Ward Identities and Fermions}
\author{L.~Del Debbio}
\date{October 2018}

\begin{document}
\maketitle

   \section{Translation Ward identity} 
    
    Find the variation of the action for the free scalar field under
    the field transformation
    \[
    \phi(x) \mapsto \phi'(x) = \phi(x) - a(x) \partial_\mu \phi(x)\, .
    \]
    Deduce the Ward identities generated by translation invariance. 

    
    
   \section{Grassmann integrals}

    Integrals over Grassmann variables are defined by specifying two
    \emph{operational} rules:
    \begin{align*}
      \int d\psi_\alpha &= 0\, ,\\
      \int d\psi_\alpha \psi_\beta &= \delta_{\alpha\beta}\, .
    \end{align*}
    Briefly discuss why this is the case.

    Show that, for an $N\times N$ matrix $A_{\alpha\beta}$ 
    \[
    \int \prod_{\beta=1}^N d\psi_\beta\,
    \prod_{\alpha=1}^N d\psibar_\alpha\,
    \exp\left(
      \psibar_\alpha A_{\alpha\beta} \psi_\beta
      \right) = \det A
    \]
    
    \emph{Hint:} it is useful to remember that 
    \[
    \det A = \sum_{\beta_1\ldots \beta_N} 
    \epsilon_{\beta_1\ldots\beta_N} 
    A_{1\beta_1} \ldots A_{N\beta_N}\, .
    \]

    

     \section{Dirac propagator} 
      
      Prove that the Dirac propagator is the inverse of the kinetic
      term in the action, \ie
      \[
      \left(i \slashed{\partial}_x -m \right)_{\alpha\beta}
      S_{\beta\gamma}(x-y) = i \delta(x-y) \delta_{\alpha\gamma}\, .
      \]

   \section{LSZ reduction for fermions}
    
    For the case of fermions the operator $\psi(x)$ can be decomposed
    as
    \[
    \psi(x) = \int d\Omega_p\, \sum_{s=\pm\nicefrac{1}{2}} \left[
      e^{-i p\cdot x} a(\mathbf{p},s) u(\mathbf{p},s) + 
      e^{i p\cdot x} b^\dagger(\mathbf{p},s) v(\mathbf{p},s)
      \right]\, .
    \]
    This relation can be inverted, yielding:
    \begin{align}
%      \label{eq:adag}
      a^\dagger(\mathbf{p},s) = \int \dd[3]{x}\, e^{-ip\cdot x}
      \bar{\psi}(x) \gamma^0 u(\mathbf{p},s) \, , \nonumber \\
%      \label{eq:bdag}
      b^\dagger(\mathbf{p},s) = \int \dd[3]{x}\, e^{-ip\cdot x}
      \bar{v}(\mathbf{p},s) \gamma^0 \psi(x) \, . \nonumber 
    \end{align}
    Following the same reasoning that we used in the case of a scalar
    field, let us introduce in the interacting theory time-dependent
    creation/annihilation operators for fermions and antifermions
    according to the expressions above. 
    Show that
    \begin{align}
      \label{eq:3}
      a^\dagger(+\infty,\mathbf{p},s) - 
      a^\dagger(-\infty,\mathbf{p},s) = 
      \int d^4x\,
      e^{-ip\cdot x} \bar\psi(x)
      (i\overleftarrow{\slashed{\partial}}+m) u(\mathbf{p},s)\, ,
    \end{align}
    and similar relations for the other creation/annihilation
    operators. 
 
    The scattering amplitude for a $2\longrightarrow 2$ process can be
    written as: 
    \begin{align}
      \label{eq:4}
      \langle p_1', s'; p_2', r';\mathrm{out} | 
      &p_1, s; p_2, r; \mathrm{in} \rangle 
      = \nonumber \\
      = \langle 0 | T\left(
        b(+\infty,\mathbf{p}_2',r') a(+\infty,\mathbf{p}_1',s') 
        a^\dagger(-\infty,\mathbf{p}_1,s) b^\dagger(-\infty,\mathbf{p}_2,r)
        \right) |0 \rangle\, .
    \end{align}
    Show that
    \begin{align}
      \label{eq:5}
      \langle p_1', s'; p_2', r';\mathrm{out} |
      & p_1, s; p_2, r; \mathrm{in} \rangle = (-i)^{2} (i)^{2} \nonumber \\
      & \times \int d^4x_1\, e^{-i p_1\cdot x_1}  
        \int d^4x_2\, e^{-i p_2\cdot x_2}
        \int d^4x_1'\, e^{i p_1'\cdot x_1'} 
        \int d^4x_2'\, e^{i k_2'\cdot x_2'} \nonumber \\
      & \times \left[\bar{u}(\mathbf{p}_1',s') 
         \left(i\overrightarrow{\slashed{\partial}}_{x_1'} -
         m\right)\right]_{\alpha_1}\,  
         \left[\bar{v}(\mathbf{p}_2,r) 
         \left(i\overrightarrow{\slashed{\partial}}_{x_2} -
         m\right)\right]_{\beta_2}   \nonumber \\
      & \times \langle 0 | T\left(
        \bar\psi_{\alpha_1}(x_1') \psi_{\alpha_1}(x_2') \bar{\psi}_{\beta_1}(x_1) \psi_{\beta_2}(x_2)
        \right) |0 \rangle \nonumber \\
      & \times \left[\left(-i\overleftarrow{\slashed{\partial}}_{x_1} -
        m\right) u(\mathbf{p}_1,s)\right]_{\beta_1}\,  
        \left[\left(-i\overleftarrow{\slashed{\partial}}_{x_2'} -
        m\right) v(\mathbf{p}_2',r')\right]_{\alpha_2}\, .        
    \end{align}


\end{document}
