\documentclass{tutorial}

\subject{Quantum Field Theory}
\subtitle{Problem Sheet 7}
\title{Loops}
\author{L.~Del Debbio}
\date{November 2018}

\begin{document}
\maketitle

\begin{enumerate}
\item \emph{ Feynman parameters} \\

  Using the exponentiation
  \[
    \frac{1}{A} = \int_0^\infty d\lambda\, e^{-\lambda A}\, ,
  \]
  and the definition of the Gamma function,
  \[
    \Gamma(z) = \int_0^\infty dt\, t^{z-1} e^{-x}\, ,
  \]
  prove the Feynman parametrization formula:
  \[
    \frac{1}{A_1^{\alpha_1} \ldots A_n^{\alpha_n}} =
    \frac{\Gamma(\alpha_1+\ldots+\alpha_n)}{\Gamma(\alpha_1) \ldots
      \Gamma(\alpha_n)} \int_0^1\dd{x_1} \ldots \dd{x_n}\,
    \delta(1-x_1-\ldots -x_n)\,
    \frac{x_1^{\alpha_1-1} \ldots x_n^{\alpha_n-1}}{\left[x_1 A_1 +
        \ldots + x_n A_n\right]^{\alpha_1+\ldots +\alpha_n}}\, .
  \]

  \bigskip
  
\item \emph{ Cheng-Wu theorem}\\

  Prove that the Feynman parametrization remains true if the
  integration
  \[
    \int_0^1dx_1 \ldots \dd{x_n}\,
    \delta(1-x_1-\ldots -x_n)
  \]
  is replaced by
  \[
    \int_0^1 \left(\prod_{l\in\nu} \dd{x_l}\right)
    \delta\left(1-\sum_{l\in\nu}x_l\right)
    \int_0^\infty \left(
      \prod_{k\not\in\nu}\dd{x_k}\, .
    \right)
  \]

  \bigskip

\item \emph{ Schwinger time representation}\\

  Using the exponentiation formula
  \[
    \frac{1}{(p^2+M^2)^a} = \frac{1}{\Gamma(a)} \int_0^\infty dT\,
    T^{a-1} e^{-T (p^2+M^2)}\, ,
  \]
  show that
  \[
    \int \frac{d^Dp}{(2\pi)^D}\, \frac{1}{p^2+M^2} =
    \frac{1}{(4\pi)^{\nicefrac{D}{2}}}  \frac{1}{\Gamma(a)} \left(M^2\right)^{\nicefrac{D}{2}-a}\, \int_0^\infty
    d\tau \tau^{a-\nicefrac{D}{2}-1} e^{-\tau}\, ,
  \]
  where $M^2$ is a generic function that does not depend on $p$, and
  the integral has already been Wick rotated to Euclidean space. \\

  Identify the divergence of the integral, and show that the UV
  divergence in the $p$ integration translates into a divergence of
  the integral over $\tau$ near the lower end of the integration
  interval. \\

  Regularize the integral by introducing a non vanishing lower
  integration limit
  \[
    \int_{T_0}^\infty dT\, T^{a-\nicefrac{D}{2}-1} e^{-T M^2}\,
  \]
  and discuss the degree of divergence of the integral. 

\item \emph{ Massless tadpoles}\\

  Consider the identity
  \[
    \int \frac{d^Dp}{(2\pi)^D}\, \frac{m^2}{p^2(p^2+m^2)} =
    \int \frac{d^Dp}{(2\pi)^D}\, \left[
      \frac{1}{p^2} - \frac{1}{(p^2+m^2)}
    \right]\, .
  \]
  Evaluate the LHS and the second term on the RHS using Schwinger's
  proper-time parametrization, and show that
  \[
    \int \frac{d^Dp}{(2\pi)^D}\, \frac{1}{p^2} = 0\, .
  \]
  This result is known as {\em Veltman's formula}.

  \bigskip

  \item \emph{ Momentum cut-off and Pauli-Villars}

    Compute
    \begin{align*}
      &\int_\Lambda \frac{d^4p}{(2\pi)^4}\, \frac{1}{p^2+m^2}\, , \\
      &\int_\Lambda \frac{d^4p}{(2\pi)^4}\,
        \frac{1}{\left[(p-k)^2+m^2\right] \left(p^2+m^2\right)}\, ,
    \end{align*}
    where the suffix indicates that the integral over the radial
    momentum coordinate is cut off at $\Lambda$.

    Remember that the solid angle in $D$ dimensions is given by
    \[
      \Omega_D = \frac{2\pi^{\nicefrac{D}{2}}}{\Gamma(\nicefrac{D}{2})}\, .
    \]
   
    The Pauli-Villars regularizaition is defined by replacing each
    propagator by a more convergent expression:
    \[
      \frac{1}{p^2+m^2} \longrightarrow
      \frac{1}{p^2+m^2} - \frac{1}{p^2+M^2} \, .
    \]
    Discuss what happens to the two integrals above when PV
    regularization is used. 
    
\end{enumerate}
\end{document}
