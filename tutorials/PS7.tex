\documentclass[12pt,a4paper]{article}
%
\setlength{\oddsidemargin}{  0mm}
\setlength{\topmargin}    { -12mm}
\setlength{\textheight}   { 246mm}
\setlength{\textwidth}    { 165mm}
\setlength{\parindent}    {  0   pt}  % not actually required but they

\setlength{\parskip}      {  6   pt}  % make paragraphs look less ugly

%

\usepackage{amsmath}
\usepackage{amssymb}
\usepackage{slashed}
\pagestyle{empty}


\usepackage[UKenglish]{babel}
\usepackage[UKenglish]{isodate}
\addto\captionsUKenglish{\renewcommand{\chaptername}{Lecture}}

\pagestyle{headings}

\usepackage{mathpazo}

\usepackage{amsmath}
\usepackage{amsfonts}
\usepackage{amssymb}
\usepackage{dsfont}
\usepackage{pifont}
%\usepackage{bbold}
\usepackage{graphicx}
\usepackage{epstopdf}
\usepackage{epsfig}
%\usepackage{bibunits}
%\usepackage{theorem}
\usepackage[framed]{ntheorem}
\usepackage{framed}
%\usepackage{showlabels}
\usepackage{makeidx}
\usepackage{simplewick}
\usepackage{tikz-feynman}
\usepackage{slashed}
\usepackage{appendix}
\usepackage[hidelinks]{hyperref}


\tikzfeynmanset{compat=1.0.0}  

\newcommand{\tick}{\ding{52}}
\newcommand{\notick}{\ding{56}}
\newcommand{\D}{\displaystyle}
\renewcommand{\chaptername}{Lecture}

\def\bfx{{\mathbf x}}
\def\bfxp{{\mathbf x^\prime}}
\def\bfy{{\mathbf y}}
\def\bfyp{{\mathbf y^\prime}}
\def\bfp{{\mathbf p}}
\def\bfpp{{\mathbf p^\prime}}
\def\ddt{\frac{d}{dt}}
\def\ddtt{\frac{d^2}{dt^2}}
\def\ie{\emph{ i.e.}\ }
\def\eg{\emph{ e.g.}\ }
\def\viz{\emph{ viz.}\ }
\def\matF{\mathcal F}
\def\matE{\mathcal E}
\def\GL{\mathrm{GL}}
\def\kpsi{|\psi\rangle}
\def\kpsione{|\psi_1\rangle}
\def\kpsitwo{|\psi_2\rangle}
\def\kpsionep{|\psi_1^\prime\rangle}
\def\kpsitwop{|\psi_2^\prime\rangle}
\def\kpsii{|\psi_i\rangle}
\def\kpsin{|\psi_n\rangle}
\def\kpsip{|\psi^\prime\rangle}
\def\bpsi{\langle\psi |}
\def\bpsione{\langle\psi_1 |}
\def\bpsitwo{\langle\psi_2 |}
\def\bpsii{\langle\psi_i |}
\def\bpsip{\langle\psi^\prime |}
\def\kphi{|\phi\rangle}
\def\kphione{|\phi_1\rangle}
\def\kphitwo{|\phi_2\rangle}
\def\kphii{|\phi_i\rangle}
\def\kphip{|\phi^\prime\rangle}
\def\bphi{\langle\phi |}
\def\bphione{\langle\phi_1 |}
\def\bphitwo{\langle\phi_2 |}
\def\bphii{\langle\phi_i |}
\def\bphip{\langle\phi^\prime |}
\def\bchi{\langle\chi |}
\def\bchione{\langle\chi_1 |}
\def\bchitwo{\langle\chi_2 |}
\def\bchii{\langle\chi_i |}
\def\bchip{\langle\chi^\prime |}
\def\kjm{|j,m\rangle}
\def\tr{\mathrm{Tr}}
\def\id{\mathds{1}}
{\theoremstyle{plain} \theorembodyfont{\rmfamily} \newframedtheorem{Ex}{Exercise}[section]}
{\theoremstyle{plain} \theorembodyfont{\rmfamily} \newtheorem{Def}{Definition}[section]}
{\theoremstyle{plain} \theorembodyfont{\rmfamily} \newtheorem{Thm}{Theorem}[section]}

\newcommand{\clearemptydoublepage}{\newpage{\pagestyle{empty}\cleardoublepage}}
\newcommand{\HRule}{\rule{\linewidth}{0.5mm}}
\newcommand{\iu}{\underline{i}}
\newcommand{\ju}{\underline{j}}
\newcommand{\ku}{\underline{k}}
\newcommand{\ru}{\underline{r}}
\newcommand{\pu}{\underline{p}}
\newcommand{\Lu}{\underline{L}}
\newcommand{\Ju}{\underline{J}}
\newcommand{\lap}{\nabla^2}
\newcommand{\ad}{\hat{a}}
\newcommand{\ac}{\hat{a}^\dagger}
\newcommand{\re}{\mathrm{Re}}
\newcommand{\ket}[1]{| #1 \rangle}
\newcommand{\bra}[1]{\langle #1 |}
\newcommand{\braket}[2]{\langle #1 | #2 \rangle}
\newcommand{\pref}[1]{(\ref{#1})}
\newcommand{\Eqref}[1]{Eq.~(\ref{#1})}
\newcommand{\del}{\v{\nabla}}				% Underlined del
\def\plusheight{-\the\dimexpr\fontdimen22\textfont2\relax}
\newcommand{\mphys}{m_\mathrm{phys}}
\newcommand{\psibar}{\bar{\psi}}
\newcommand{\etabar}{\bar{\eta}}
\newcommand{\munu}{{\mu\nu}}
\newcommand{\tDelta}{\tilde{\Delta}}
\newcommand{\SProp}[1]{\frac{1}{#1^2-m^2+i\epsilon}}
\newcommand{\ESProp}[2]{\frac{1}{\left(#1^2+m^2\right)^#2}}

\newcommand{\tphi}{\tilde{\phi}}
\newcommand{\tj}{\tilde{J}}
\newcommand{\tchi}{\tilde{\chi}}
\newcommand{\psibar}{\bar{\psi}}
\newcommand{\etabar}{\bar{\eta}}
\newcommand{\munu}{{\mu\nu}}

\begin{document}
\begin{center}
{\bf Quantum Field Theory}\\[\baselineskip]
\end{center}
{\bf Problem Sheet 7}

\begin{enumerate}
\item {\it Feynman parameters} \\

  Using the exponentiation
  \[
    \frac{1}{A} = \int_0^\infty d\lambda\, e^{-\lambda A}\, ,
  \]
  and the definition of the Gamma function,
  \[
    \Gamma(z) = \int_0^\infty dt\, t^{z-1} e^{-x}\, ,
  \]
  prove the Feynman parametrization formula:
  \[
    \frac{1}{A_1^{\alpha_1} \ldots A_n^{\alpha_n}} =
    \frac{\Gamma(\alpha_1+\ldots+\alpha_n)}{\Gamma(\alpha_1) \ldots
      \Gamma(\alpha_n)} \int_0^1dx_1 \ldots dx_n\,
    \delta(1-x_1-\ldots -x_n)\,
    \frac{x_1^{\alpha_1-1} \ldots x_n^{\alpha_n-1}}{\left[x_1 A_1 +
        \ldots + x_n A_n\right]^{\alpha_1+\ldots +\alpha_n}}\, .
  \]

  \bigskip
  
\item {\it Cheng-Wu theorem}\\

  Prove that the Feynman parametrization remains true if the
  integration
  \[
    \int_0^1dx_1 \ldots dx_n\,
    \delta(1-x_1-\ldots -x_n)
  \]
  is replaced by
  \[
    \int_0^1 \left(\prod_{l\in\nu} dx_l\right)
    \delta\left(1-\sum_{l\in\nu}x_l\right)
    \int_0^\infty \left(
      \prod_{k\not\in\nu}dx_k\, .
    \right)
  \]

  \bigskip

\item {\it Schwinger time representation}\\

  Using the exponentiation formula
  \[
    \frac{1}{(p^2+M^2)^a} = \frac{1}{\Gamma(a)} \int_0^\infty dT\,
    T^{a-1} e^{-T (p^2+M^2)}\, ,
  \]
  show that
  \[
    \int \frac{d^Dp}{(2\pi)^D}\, \frac{1}{p^2+M^2} =
    \frac{1}{(4\pi)^{D/2}}  \frac{1}{\Gamma(a)} \left(M^2\right)^{D/2-a}\, \int_0^\infty
    d\tau \tau^{a-D/2-1} e^{-\tau}\, ,
  \]
  where $M^2$ is a generic function that does not depend on $p$, and
  the integral has already been Wick rotated to Euclidean space. \\

  Identify the divergence of the integral, and show that the UV
  divergence in the $p$ integration translates into a divergence of
  the integral over $\tau$ near the lower end of the integration
  interval. \\

  Regularize the integral by introducing a non vanishing lower
  integration limit
  \[
    \int_{T_0}^\infty dT\, T^{a-D/2-1} e^{-T M^2}\,
  \]
  and discuss the degree of divergence of the integral. 

\item {\it Massless tadpoles}\\

  Consider the identity
  \[
    \int \frac{d^Dp}{(2\pi)^D}\, \frac{m^2}{p^2(p^2+m^2)} =
    \int \frac{d^Dp}{(2\pi)^D}\, \left[
      \frac{1}{p^2} - \frac{1}{(p^2+m^2)}
    \right]\, .
  \]
  Evaluate the LHS and the second term on the RHS using Schwinger's
  proper-time parametrization, and show that
  \[
    \int \frac{d^Dp}{(2\pi)^D}\, \frac{1}{p^2} = 0\, .
  \]
  This result is known as {\em Veltman's formula}.

  \bigskip

  \item {\it Momentum cut-off and Pauli-Villars}

    Compute
    \begin{align*}
      &\int_\Lambda \frac{d^4p}{(2\pi)^4}\, \frac{1}{p^2+m^2}\, , \\
      &\int_\Lambda \frac{d^4p}{(2\pi)^4}\,
        \frac{1}{\left[(p-k)^2+m^2\right] \left(p^2+m^2\right)}\, ,
    \end{align*}
    where the suffix indicates that the integral over the radial
    momentum coordinate is cut off at $\Lambda$.

    Remember that the solid angle in $D$ dimensions is given by
    \[
      \Omega_D = \frac{2\pi^{D/2}}{\Gamma(D/2)}\, .
    \]
   
    The Pauli-Villars regularizaition is defined by replacing each
    propagator by a more convergent expression:
    \[
      \frac{1}{p^2+m^2} \longrightarrow
      \frac{1}{p^2+m^2} - \frac{1}{p^2+M^2} \, .
    \]
    Discuss what happens to the two integrals above when PV
    regularization is used. 
    
\end{enumerate}

\vfill
\hspace*{\fill}\tiny L Del Debbio, November 2018.
\end{document}
