\documentclass{tutorial}

\subject{Quantum Field Theory}
\subtitle{Problem Sheet 1}
\title{Gaussian Integrals}
\author{L.~Del Debbio}
\date{September 2018}

\begin{document}
\maketitle

\begin{enumerate}

\item The general Gaussian integral can be readily evaluated:
  \begin{align}
    Z_A(b) &= \int d^nx\, \exp\left(
             -\frac12 \sum_{i,j=1}^n x_i A_{ij} x_j
             + \sum_{i=1}^n b_i x_i
             \right) \\
    \label{eq:GaussIntThree}
           &=  \left(2\pi\right)^{n/2} \left(\det A \right)^{-1/2}
             \exp\left(
             \frac12 \sum_{i,j=1}^n b_i \Delta_{ij} b_j
             \right) \, ,
  \end{align}
  where $\Delta = A^{-1}$. The existence of $\Delta$ is guaranteed by
  the non-vanishing e.vals of $A$.

  Check the result in Eq.~\ref{eq:GaussIntThree} by changing the
  integration variables in the integral:
  \[
    x_i = y_i + \sum_{j=1}^n \Delta_{ij} b_j\, .
  \]

\item Let $x$ and $y$ be stochastic variables such that:
  \begin{align*}
    \langle x \rangle = \langle y \rangle =0\, ,
    \langle x^2 \rangle = 5\, ,
    \langle x y \rangle = 3 \, ,
    \langle y^2 \rangle = 2\, ,
  \end{align*}
  compute $\langle x^4\rangle$, $\langle x^3 y \rangle$, $\langle x^2
  y^2\rangle$.  

\item Compute the six-point function
  \[
    \langle x_{i_1} x_{i_2} x_{i_3} x_{i_4} x_{i_5} x_{i_6} \rangle    
  \]
  using Wick's theorem. 

\item Consider 
  \begin{align}
    \langle x_k F(x) \rangle_0 = 
    \int d\mu_0(x) x_k F(x) 
  \end{align}
  Show that
  \begin{align}
    \langle x_k F(x) \rangle_0  = 
    \sum_l \langle x_k x_l \rangle_0 \langle \frac{\partial F}{\partial x_l}\rangle_0\, .
  \end{align}
  {\em Hint:} 
  \begin{align}
    - \sum_l \Delta_{kl} \frac{\partial}{\partial x_l}
    \exp\left(
    -\frac12 \sum_{i,j=1}^n x_i A_{ij} x_j
    \right) = 
    x_k \exp\left(
    -\frac12 \sum_{i,j=1}^n x_i A_{ij} x_j
    \right)\, .
  \end{align}

\item   Compute the ratio 
\begin{align}
  Z(\lambda)/Z(0)
\end{align}
for the potential 
\begin{equation}
  \label{eq:QuarticPot}
  V(x) = \frac{1}{4!} \sum_{i=1}^n x_i^4\, ,
\end{equation}
to second order in $\lambda$.

\item Show that all the vacuum contributions cancel when computing
  $\langle x_{i_1} x_{i_2}\rangle$. The final result is
  \begin{align}
    \frac{1}{Z(\lambda)} \int d^nx\, e^{-S(x;\lambda)} 
    x_{i_1} x_{i_2} = & 
                             \left[
                             \Delta_{i_1 i_2} - \lambda  \left(\frac12 \sum_{i=1}^n
                             \Delta_{i i_1} \Delta_{i i_2} \Delta_{i i}\right) 
                             + \right. \nonumber \\
                           & \left.
                             + \lambda^2 \left( 
                             \frac{1}{4} \sum_{i,j=1}^n \Delta_{i i_1}
                             \Delta_{i i_2} \Delta_{i j}^2 \Delta_{jj}
                             + \frac{1}{6} \sum_{i,j=1}^n \Delta_{i i_1}
                             \Delta_{j i_2} \Delta_{i j}^3 +
                             \right. \right. \nonumber \\
    \label{eq:TwoPointNorm}
                           & + \left. \left.
                             \frac{1}{4} \sum_{i,j=1}^n \Delta_{i i_1}
                             \Delta_{j i_2} \Delta_{i j} \Delta_{ii}\Delta_{jj}
                             \right)
                             \right]\, .
  \end{align}
  Write a diagrammatic representation for the contributions $O(\lambda^2)$.

\end{enumerate}
\end{document}
