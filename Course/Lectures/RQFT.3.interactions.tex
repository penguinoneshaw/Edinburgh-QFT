% !TeX root = ../notes.tex

\documentclass[notes.tex]{subfiles}


\begin{document}
\chapter{Interactions and Feynman Diagrams}
\label{chap:rqft_interactions}
Thus far, we've only considered the `free field' Lagrangian, entirely quadratic in the fields. Quantisation gave us a field of independent harmonic oscillators, equivalent to entirely non-interacting particles.

This isn't very interesting.

To make it interesting, we add interactions. For scalar fields, the simplest interactions are in what is referred to as $\phi^3$ theory, which is simply where we add \[\mathcal{L}_I = -\frac{1}{6} g \phi^3\] to the free Lagrangian $\mathcal{L}_0$ to get
\begin{align*}
	\mathcal{L} & \equiv \mathcal{L}_0 + \mathcal{L}_I                                               \\
	            & = \half(\partial^\mu\phi)(\partial_\mu\phi) - \half m^2\phi^2 - \frac{1}{6}g\phi^3
\end{align*}
where $g$ is referred to as the `coupling constant'.
The Euler-Lagrange equation gives, predictably, that
\begin{align*}
	(\partial^2 - m^2)\phi & = -\half{}g\phi^2
\end{align*}
which is notably non-linear, which is always true for interesting fields.
Non-linear of course also means more difficult to solve, which is a problem, but we try our best by working in perturbation theory and requiring $g\ll m$ so the interaction is small.

\section{Refining the Green's Function}
Classically, we use a Green's function
\begin{align*}
	(\partial^2 + m^2)G(x) & = \delta^4(x)
\end{align*}
which we can find using the Fourier representation
\begin{align*}
	G(x) & = -\int\frac{\dd[4]{p}}{(2\pi)^4} \frac{e^{-p\vdot{x}}}{p^2 - m^2}
\end{align*}
to get the integral version.
However, a Green's function is only well defined in conjunction with boundary conditions, which we must choose to suit the system being solved.
Classically we use the `retarded' boundary condition,
\begin{align*}
	G_R(x) & = \Theta(t)\Delta(x)
\end{align*}
then if
\begin{align*}
	(\partial^2 + m^2)\phi & = j,                                                           \\
	\phi(x)                & = \phi_0(x) + \int \dd[4]{x^\prime} G_R(x-x^\prime)j(x^\prime) \\
	                       & = \phi_0 - \half g\int\phi^2(x^\prime)G_R(x-x^\prime)
\end{align*}
which leads us to expand $\phi$ as a Taylor series to get
\begin{align*}
	\phi & \approx \phi_0 -\half g \int\dd[4]{x} G_R(x-x^\prime) \phi_0(x) + \mathcal{O}(g^2)
\end{align*}
which is a perturbative solution for the classical field.

\section{Interaction or `Dirac' Picture}
For perturbation purposes, it helps to have the `interaction' picture, which for the free Hamiltonian is the same as the Heisenberg.
The difference is that the Dirac picture always approximates the Hamiltonian for the time evolution operator with the free Hamiltonian.

That is to say, if we write
\begin{align*}
	H & = H_0 + H_I
\end{align*}
then in the interaction picture
\begin{align*}
	\ket{\psi, t}_D & = e^{iH_0 t}\ket{\psi, t}_S      & A_D(t) & = e^{iH_0t}A_Se^{-iH_0t} \\
	                & = e^{iH_0t} e^{-iHt}\ket{\psi}_H
\end{align*}
which allows us to rewrite the time evolution as
\begin{align*}
	i\pdv{t}\ket{\psi, t}_D & = -H_0e^{iH_0t}\ket{\psi, t}_S + e^{iH_0t}\pdv{t}\ket{\psi, t}_S \\
	                        & = e^{iH_0t}H_I^S\ket{\psi, t}_S                                  \\
	                        & = H_I^D\ket{\psi,t}_D
\end{align*}
so states evolve in time due to the interaction, but not any other factors, in the same framework as in the Heisenberg picture with
\begin{align*}
	\pdv{T}A_D(t) & = i\commutator{H_0}{A_D(t)}.
\end{align*}

Note that \(H_0^D = H_0^S\) by design and $H^H = H^S$, but $H^D_I \neq H^S_I$ because $\commutator{H_0}{H_I}\neq 0$. In the limit of no interaction, the Dirac picture obviously returns to the Heisenberg.

\section{The S Matrix}
A scattering process is defined as a process having some initial state of `free' particles with a definite set of good quantum numbers (momentum, energy, spin, etc.) at $t=-\infty$ (`the past').
We can then describe the scatter with the simple trasition
\begin{align*}
	\ket{\Psi, \infty} & = S\ket{\Psi, -\infty}
\end{align*}
where $S$ contains and represents all the interesting physics and is called the S-matrix or scattering operator.

Consider scattering with an initial state $\ket{\Psi, -\infty} = \ket{i}$ and a complete basis $\{\ket{f}\}$ of possible final states such that
\begin{align*}
	\ket{\Psi, \infty} & = \sum_f \ket{f}\braket{f}{\Psi, \infty} \\
	                   & = \sum_f \ket{f} \mel{f}{S}{i}
\end{align*}
which means that \(
\abs{\braket{f}{\Psi, \infty}}^2 = \abs{\mel{f}{S}{i}}^2
\)
is the probability of finding the system in the final state $\ket{f}$.
This means that the S-matrix elements $S_{fi}$ are the probability amplitudes for transitions from $i$ to $f$. We can also show that $S$ is unitary, which is good, as that implies conservation of probability.

Scattering in QFT is more general than in non-relativistic quantum mechanics as it is able to be inelastic --- particles can be created and destroyed, not just interact with each other.

\section{Dyson Series}
To find a formal solution to
\begin{align*}
	i\pdv{t}\ket{\psi, t} & = H_I \ket{\psi, t}
\end{align*}
we must iterate on the integral solution, which gives
\begin{align*}
	\ket{\psi, t} & = \ket{\psi, -\infty} - i\int^t_{-\infty}\kern-0.5em \dd{t_1} H_I(t_1)\ket{\psi, t}                                                                                                                                \\
	              & = \ket{\psi, -\infty} - i\int^t_{-\infty}\kern-0.5em \dd{t_1} H_I(t_1)\qty[\ket{\psi, -\infty} - i\int^{t_1}_{-\infty}\kern-0.5em \dd{t_2} H_I(t_2)\ket{\psi, t}]                                                  \\
	              & = \qty[\sum_{n=0}^\infty (-i)^n \int_{-\infty}^\infty\kern-0.5em \dd{t_1}\int^{t_1}_{-\infty}\kern-0.5em \dd{t_2} \cdots \int^{t_{n-1}}_{-\infty}\kern-0.5em \dd{t_n}H_I(t_1)\cdots H_I(t_n)]\ket{\psi, -\infty} .
\end{align*}
The format of this can be simplified somewhat by using the time ordered product symbol
\begin{align*}
	T\qty(A(t_1)B(t_2)) & = A(t_1)B(t_2) \Theta(t_1-t_2) +B(t_2) A(t_1) \Theta(t_2 - t_1)
\end{align*}
which means that we can make all of the integrals over all time to get
\begin{align*}
	S & = \sum_{n=0}^\infty \frac{(-i)^n}{n!} \int^{\infty}_{-\infty}\kern-0.5em \dd{t_1}\int^{\infty}_{-\infty}\kern-0.5em \dd{t_2} \cdots \int^{\infty}_{-\infty}\kern-0.5em \dd{t_n}T\qty(H_I(t_1)\cdots H_I(t_n)) \\
	  & = \sum_{n=0}^\infty \frac{(-i)^n}{n!} \int \dd[4]{x_1}\int \dd[4]{x_2} \cdots \int \dd[4]{x_n}T\qty(\mathcal{H}_I(x_1)\cdots \mathcal{H}_I(x_n))
\end{align*}
which is good, as it at least on the surface looks Lorentz invariant.

Note that for non-derivative interaction Lagrangians,
\begin{align*}
	\mathcal{H}_I & = -\mathcal{L}_I                      &
	S             & = T\exp[i\int\dd[4]{x} \mathcal{L}_I]
\end{align*}

For a given initial and final state, we need to find terms in $\mathcal{L}_I$ which contribute to $\ket{i}\to\ket{f}$:
\begin{itemize}
	\item Annihilation and creation operators to destroy initial and create final on-shell particles;
	\item Additional operators to create and destroy intermediate `virtual' particles, which are generally not on-shell.
\end{itemize}
In the interaction picture, fields satisfy the free-field equations, i.e. \(a\) and \(a^\dagger\) are all the same.
If we use normal ordering, we always destroy the incoming and create the outgoing.

\begin{example}
	In $\phi^3$ theory, $\mathcal{H}_I = -\mathcal{L}_I = \frac{1}{6}g_i\phi^3$.
	We remember that $\phi(x) = \phi^+(x) + \phi^-(x)$, so we have an object $\phi^3$ which is the sum of the combinations of products of $\phi^\pm$. We need an equal number of incoming and outgoing particles in order to conserve momentum, but this is not possible to first order in $g$ because of the structure of the Dyson series.
	We need \emph{at least} two $\mathcal{L}_I$ and thus have
	\begin{align*}
		g^2 \phi^-\phi^-\phi^+\phi^-\phi^+\phi^+ & = \begin{tikzpicture}[baseline={(current bounding box.center)}]
			\begin{feynman}
				\vertex (v1);
				\vertex [right=of v1](v2);
				\vertex [above left=of v1] (i1);
                \vertex [below left=of v1](i2);
				\vertex [below right=of v2](f2);
				\vertex [above right=of v2](f1);
				\diagram {
                    (i1) -- (v1),
                    (i2) -- (v1),
                    (v1) -- (v2),
                    (v2) -- (f1),
                    (v2) -- (f2)
				};
				\vertex [below=0.2em of v1] {\(_{t_1}\)};
				\vertex [below=0.2em of v2] {\(_{t_2}\)};
			\end{feynman}
		\end{tikzpicture}
	\end{align*}
\end{example}

We also want normal ordering, but in the Dyson expansion, we have time ordering.
Consider
\begin{align*}
	A&=A^+ + A^- & B = B^+ + B^-
\end{align*}
which means
\begin{align*}
	A(x)B(x^\prime) - \normalorder{A(x)B(x^\prime)} &= \commutator{A^+(x)}{B^-(x^\prime)}
\end{align*}
but 
\begin{align*}
	\commutator{A^+(x)}{B^-(x^\prime)} &= \mel{0}{\commutator{A^+(x)}{B^-(x^\prime)}}{0}\\
	&= \mel{0}{A(x) B(x^\prime)}{0}
\end{align*}
because the commutator is just a complex number, so 
\begin{align*}
	A(x) B(x^\prime) &= \normalorder{A(x)B(x^\prime)} + \mel{0}{A(x) B(x^\prime)}{0}.
\end{align*}

Now we try the same thing with time ordered products such that
\begin{align*}
	T\qty[A(x)B(x^\prime)] &= \Theta\qty(x^0 - (x^\prime)^0) A(x)B(x^\prime) + \Theta\qty((x^\prime)^0 - x^0)B(x^\prime)A(x)
\end{align*}
and since
\begin{align*}
	\normalorder{A(x)B(x^\prime)}&= \normalorder{B(x^\prime)A(x)}\intertext{we see that}
	\normalorder{T[A(x)B(x^\prime)]}&=\normalorder{A(x)B(x^\prime)}
\end{align*}
which allows us to finally write
\begin{align*}
	\underbrace{T\qty[A(x)B(x^\prime)]}_\text{Appears in Dyson expansion} &= \normalorder{A(x)B(x^\prime)} + \underbrace{\mel{0}{T\qty[A(x)B(x^\prime)]}{0}}_{(*)}
\end{align*}
The expression $(*)$ is the interesting bit. It is denoted $\contraction{}{A}{(x)}{B}A(x)B(x^\prime)$ and called the contraction. This gives us a canonical solution to the Dyson expansion but needs to be generalised to $n$ operators, which we extend in the logical way, such that
\begin{align*}
	\normalorder{\contraction{A}{B}{}{C}\contraction{ABCD}{E}{\cdots{}K}{L}\contraction[2ex]{}{A}{BCDE\cdots{}}{K}ABCDE\cdots{}KLM\cdots} &= \contraction{}{A}{}{K}AK\contraction{}{B}{}{C}BC\contraction{}{E}{}{L}EL\normalorder{DFGHIJM\ldots}.
\end{align*}

\subsection{Wick's Theorem}
Wick's theorem gives us a a formula for the time ordered product of operators at unequal times in terms of normal ordered products and contractions, which is
\begin{align*}
	T\qty[ABCD\ldots] &= \normalorder{ABCD\ldots}\\
	&\quad + \normalorder{\contraction{}{A}{}{B}ABCD\ldots} + \normalorder{\contraction{}{A}{B}{C}ABCD\ldots} + \cdots + \normalorder{\ldots\contraction{WX}{Y}{}{Z}WXYZ}\\
	&\quad + \normalorder{\contraction{}{A}{}{B}AB\contraction{}{C}{}{D}CD\ldots} + \cdots
\end{align*}
where the right hand side is the sum of the normal ordered products of all combinations of increasing numbers of contractions.

In the S-matrix expansion, we have `mixed' time ordered products with operators at equal and unequal times such that
\begin{align*}
	T\qty[\mathcal{H}_I(x_1) \cdots \mathcal{H}_I(x_n)] &= T\qty[\normalorder{A(x_1)B(x_1)}\cdots\normalorder{A(x_n)B(x_n)}]
\end{align*}
which Wick's theorem can be extended to cover by utilising that 
\begin{align*}
	T\qty[\normalorder{ABC}] &= \normalorder{ABC}
\end{align*}
So the only non-zero contractions are from fields at different times.

\section{The Feynman Propagator}\label{sec:rqft_interactions_feynman-prop}
Wick's Theorem on the right hand side has two ingredients
\begin{itemize}
	\item Normal ordered products
	\item Contractions
\end{itemize}
which means that for a scalar field, the contractions are
\begin{align*}
	\contraction{}{\phi}{(x)}{\phi}\phi(x)\phi(x^\prime) &= \mel{0}{T\qty[\phi(x)\phi(x^\prime)]}{0}\\
	&= \Theta(t-t^\prime)\expval{\phi(x)\phi(x^\prime)}{0} + \Theta(t^\prime-t)\expval{\phi(x^\prime)\phi(x)}{0}\\
	&= i\Theta(t-t^\prime)\Delta^+(x-x^\prime) - i\Theta(t-t^\prime)\Delta^-(x-x^\prime)
\end{align*}
which we assign the new symbol $\Delta_F(x-x^\prime)$ and call the Feynman Propagator.
This handily automatically captures the time ordering in the Dyson series, but it also means that negative energy modes are necessary for the system to be consistent.

A useful representation of $i\Delta_F(x)$ is 
\begin{align*}
	\Delta_F(x) &= \int_{C_F} \frac{\dd[4]{p}}{(2\pi)^4}\qty[\frac{e^{-ip\vdot{x}}}{p^2 - m^2}]
\end{align*}
where $C_F$ is a contour in the $p_0$ plane which passes along the real axis, below the $p^0 = -\omega(p)$ and above the $p^0=\omega(p)$ poles which we then complete at infinity such that 
\begin{itemize}
	\item $x_0>0$ we include the lower half-plane, which means that we select the positive energy pole.
	The residue at $p_0 = \omega(\vec{p})$ is $-\Delta^+(x)$.
	\item $x_0 < 0$ means that we need to close on the positive half-plane in $p^0$ space (in order to exponentially suppress the real part of the integral). This gives us the pole at the negative energy solution with residue $\Delta^-(x)$.
\end{itemize}

The so-called $i\epsilon$ prescription makes this easier to deal with in a mechanical integration sense, as it pushes the contour to being entirely on the real axis.
This gives the adjusted integral
\begin{align*}
	\Delta_F(x) &= \int\frac{\dd[4]{p}}{(2\pi)^4} \qty[\frac{e^{-ip\vdot{x}}}{p^2 - m^2 + i\epsilon}]
\end{align*}
for some infinitesimal $\epsilon$ which we can then take the zero limit of. 
The poles of the new integrand are at the roots of $p_0^2 - (\omega(p) - i\nicefrac{\epsilon}{2\omega})^2$.

We note that $\Delta_F(x)$ is a Green's function for the Klein-Gordon equation (as all of its components also are), but for the Dirichlet boundary conditions which define quantum mechanics, for which we know the initial and final states of $\varphi$, but by definition cannot know $\dot\varphi$ to precise accuracy.

\section{Feynman Diagrams}
We revist the Dyson expansion and define that
\begin{align*}
	S&=\sum_{n=0}^\infty S^{(n)} & S^{(n)} &= \frac{(-i)^n}{n!} \int\dd[4]{x_1}\cdots\dd[4]{x_n} T\qty[\mathcal{H}_I(x_1)\cdots \mathcal{H}_I(x_n)]
\end{align*}
which means that $n$ labels the number of vertices in the interaction diagrams.
In Feynman diagrams, we usually have a `time axis', which in these cases runs from left to right.
For example, in scalar $\phi^3$ theory with interaction Hamiltonian
\begin{align*}
	\mathcal{H}_I (x) &= \frac{g}{3!} \normalorder{\phi^3(x)}\\
	&= \frac{g}{3!}\normalorder{(\phi^+ + \phi^-)^3}
\end{align*}
we have that
\begin{align*}
	S^{(0)} &= 1\intertext{in which nothing happens, being identical to the free particle solution, and}
	S^{(1)} &= -i\frac{g}{3!}\int \dd[4]{x}\qty((\phi^+)^3 + 3\phi^-(\phi^+)^2 + 3(\phi^-)^2\phi^+ + (\phi^-)^3)\\
	&= \begin{tikzpicture}[baseline={(current bounding box.center)}]
		\begin{feynman}
			\vertex (v1);
			\vertex [above right=1.141em of v1] (f1);
			\vertex [below right=1.141em of v1] (f2);
			\vertex [right=1em of v1] (f3);
			\diagram {
				(v1) -- (f1);
				(v1) -- (f2);
				(v1) -- (f3);
			};
		\end{feynman}
	\end{tikzpicture}
	\quad +\quad 
	\underbrace{
	\begin{tikzpicture}[baseline={(current bounding box.center)}]
		\begin{feynman}
			\vertex (v1);
			\vertex [above left=1.141em of v1] (f1);
			\vertex [below left=1.141em of v1] (f2);
			\vertex [right=1em of v1] (f3);
			\diagram {
				(f1) -- (v1);
				(f2) -- (v1);
				(v1) -- (f3);
			};
		\end{feynman}
	\end{tikzpicture}
	\quad +\quad 
	\begin{tikzpicture}[baseline={(current bounding box.center)}]
		\begin{feynman}
			\vertex (v1);
			\vertex [above right=1.141em of v1] (f1);
			\vertex [below right=1.141em of v1] (f2);
			\vertex [left=1em of v1] (f3);
			\diagram {
				(f1) -- (v1);
				(f2) -- (v1);
				(v1) -- (f3);
			};
		\end{feynman}
	\end{tikzpicture}
	}_{(*)} \quad +\quad 
	\begin{tikzpicture}[baseline={(current bounding box.center)}]
		\begin{feynman}
			\vertex (v1);
			\vertex [above left=1.141em of v1] (f1);
			\vertex [below left=1.141em of v1] (f2);
			\vertex [left=1em of v1] (f3);
			\diagram {
				(v1) -- (f1);
				(v1) -- (f2);
				(v1) -- (f3);
			};
		\end{feynman}
	\end{tikzpicture}
\end{align*}
where the first and third terms are zero, and (*) are the `interesting' but still physically unviable terms because they violate conservation of momentum --- we only have one type of field with a single mass, so we can only create or destroy quanta with that mass.

Thus in $\phi^3$ theory we must go to second order to find non-zero viable terms. We subdivide the $S$ matrix further such that we introduce a new index $i$ ($S^{(n)}_i$) which gives the number of contracted terms resulting from Wick's theorem.
At its second order,
\begin{align*}
	S^{(2)} &= -\frac{g^2}{2!(3!)^2} \int\dd[4]{x}\int\dd[4]{y} T\qty[\normalorder{\phi^3(x)}\normalorder{\phi^3(y)}]
\end{align*}
which we then split into terms with zero, one, two and three contractions. Going through these, we get that:
\begin{itemize}
	\item $S_0^{(2)}$ give us $\normalorder{\phi^3(x)\phi^3(y)}$ which is equivalent to two disconnected $S^{(1)}$. Generally, we ignore disconnected diagrams, even if nonzero, since their contribution is equal to lower orders.
	\item $S_1^{(2)}$ gives us $3^2\contraction{}{\phi}{(x)}{\phi}\phi(x)\phi(y)\normalorder{\phi^2(x)\phi^2(y)}$ (the constant coming from the fact there are three ways to pick each $\phi$). These are called `tree-level' diagrams, with one propagating virtual particle and four external particles.

\begin{figure}[htbp]
	\centering
	\begin{subfigure}{0.3\textwidth}
		\centering
		\begin{tikzpicture}[baseline={(current bounding box.center)}]
			\begin{feynman}
				\vertex [label=\(x\)](v1);
				\vertex [above left=of v1] (i1);
				\vertex [below left=of v1] (i2);
				\vertex [right=of v1,label=\(y\)] (v2);
				\vertex [above right=of v2] (f1);
				\vertex [below right=of v2] (f2);
				\diagram {
					(i1) -- (v1) -- (v2) -- (f1);
					(i2) -- (v1);
					(v2) -- (f2);
				};
			\end{feynman}
		\end{tikzpicture}
		\caption{S-channel}
	\end{subfigure}
	\begin{subfigure}{0.3\textwidth}
		\centering
		\begin{tikzpicture}[baseline={(current bounding box.center)}]
			\begin{feynman}
				\vertex [label=\(x\)](v1);
				\vertex [left=of v1] (i1);
				\vertex [right=of v1] (f1);
				\vertex [below=of v1,label=below:\(y\)] (v2);
				\vertex [left=of v2] (i2);
				\vertex [right=of v2] (f2);
				\diagram {
					(i1) -- (v1) -- (f1);
					(i2) -- (v2) -- (f2);
					(v1) -- (v2);
				};
			\end{feynman}
		\end{tikzpicture}
		\caption{T-channel}
	\end{subfigure}
	\begin{subfigure}{0.3\textwidth}
		\centering
		\begin{tikzpicture}[baseline={(current bounding box.center)}]
			\begin{feynman}
				\vertex [label=\(x\)](v1);
				\vertex [below=of v1,label=below:\(y\)] (v2);
				\vertex [left=of v1] (i1);
				\vertex [left=of v2] (i2);
				\vertex [right=of v2] (f1);
				\vertex [right=of v1] (f2);
				\diagram {
					(i1) -- (v1) -- (f1);
					(i2) -- (v2) -- (f2);
					(v1) -- (v2);
				};
			\end{feynman}
		\end{tikzpicture}
		\caption{U-channel}
	\end{subfigure}
	\caption{$S_1^{(2)}$ diagrams}
	\label{fig:rqft_interactions_tree-level}
\end{figure}

\item $S_2^{(2)}$ gives us $3!\contraction{}{\phi}{(x)}{\phi}\phi(x)\phi(y)\contraction{}{\phi}{(x)}{\phi}\phi(x)\phi(y)\,\normalorder{\phi(x)\phi(y)}$ which represents one incoming and outgoing particle, and two virtual propagating particles. These are divergent, and commonly appear as loop corrections along otherwise regular propagators.

\begin{figure}[htbp]
	\centering
	\begin{tikzpicture}[baseline={(current bounding box.center)}]
		\begin{feynman}
			\vertex (i);
			\vertex [right=of i, label=\(x\)](v1);
			\vertex [right=of v1, label=\(y\)](v2);
			\vertex [right=of v2](f);

			\diagram {
				(i) -- (v1) -- [bend right] (v2) -- (f);
				(v1) -- [bend left] (v2);
			};
		\end{feynman}
	\end{tikzpicture}
	\caption{$S_1^{(2)}$: One Loop Process}
	\label{fig:rqft_interactions_one-loop}
\end{figure}
\item $S_2^{(3)}$ gives one term, the vacuum bubble, which is simply \((3!) \contraction{}{\phi}{(x)}{\phi}\phi(x)\phi(y)\contraction{}{\phi}{(x)}{\phi}\phi(x)\phi(y) \contraction{}{\phi}{(x)}{\phi}\phi(x)\phi(y)\). It has two loops, and is also divergent, but can be ignored as its contributions are entirely imaginary.
\begin{figure}[htbp]
	\centering
	\begin{tikzpicture}[baseline={(current bounding box.center)}]
		\begin{feynman}
			\vertex [label=\(x\)](v1);
			\vertex [right=of v1, label=\(y\)](v2);
	
			\diagram {
				(v1) -- [bend right] (v2);
				(v1) -- (v2);
				(v1) -- [bend left] (v2);
			};
		\end{feynman}
	\end{tikzpicture}
	
	\caption{$S_2^{(3)}$ Vacuum Bubble}
	\label{fig:rqft_interactions_vacuum-bubble}
\end{figure}
\end{itemize}

\section{S-matrix Elements in Momentum Space}
In reality, we are usually only interested in $S_{fi} = \mel{f}{S}{i}$ rather than $S$ itself.

As we have seen, $\ket{i}$ and $\ket{f}$ are expressed in the Fock space (in terms of $a^\dagger(\vec{p})$), which means from an \emph{uncertainty principle} perspective, we know every quantum's definite momentum $\vec{p}$, so we should represent the field transitions in momentum space.

Starting from
\begin{align*}
	\ket{\vec{p}} &= a^\dagger(\vec{p}) \ket{0}
\end{align*}
and equations \ref{eqn:rqft_canonical-quantisation_creation-annihilation-fields} defining $\phi^\pm$, we find that
\begin{align*}
	\phi^+(x)\ket{\vec{p}} &= \int \dds{p^\prime} e^{-ip^\prime \vdot x}a(\vec{p}^\prime) a^\dagger(\vec{p}) \ket{0}\\
	&= \int \dds{p^\prime} e^{-ip^\prime \vdot x} \deltaslash^3(\vec{p}^\prime - \vec{p}) \ket{0}\\
	&= e^{-ip\vdot x}\ket{0}
\end{align*}
and
\begin{align*}
	\bra{\vec{p}} \phi^-(x) &= \bra{0} \int \dds{p^\prime} e^{-ip^\prime \vdot x} a(\vec{p}) a^\dagger(\vec{p}^\prime) \ket{0}\\
	&= \bra{0} e^{ip\vdot{}x}.
\end{align*}

The momentum space propagator is therefore
\begin{align*}
	\contraction{}{\phi}{(x)}{\phi}\phi(x)\phi(y) &= i\Delta_F(x-y)\\
	&= \frac{i}{(2\phi)^4} \int \dd[4]{k} \frac{e^{-k\vdot{}(x-y)}}{k^2 - m^2 + i\epsilon}. 
\end{align*}

\subsection{$\phi^3$ in Momentum Space}
At $\mathcal{O}(g)$ for the process where we have
\begin{align*}
	\ket{i} &= \ket{p}\\
	\ket{f} &= \ket{q, q^\prime}\\
	&= \ket{q}\ket{q^\prime}
\end{align*}
we see that
\begin{align*}
	\mel{f}{S_1}{i} &= -\frac{ig}{\cancel{3}\times 2}\mel**{q, q^\prime}{\,\cancel{3}\!\int \dd[4]{x} (\phi^-(x))^2 \phi^+(x)}{p}\\
	&= -ig \int \dd[4]{x} e^{i(q+q^\prime) \vdot x} e^{ip\vdot{}x}\braket{0}{0}\\
	&= -ig(2\pi)^4 \delta^4(q+q^\prime - p).
\end{align*}
\begin{figure}[htbp]
	\centering
	\begin{tikzpicture}[baseline={(current bounding box.center)}]
		\begin{feynman}
			\vertex [label=below:\(x\)] (v1);
			\vertex [above right= of v1,label=right:\(q^\prime\)] (f1);
			\vertex [below right= of v1,label=right:\(q\)] (f2);
			\vertex [left= of v1,label=left:\(p\)] (i);
			\diagram {
				(i) --  (v1);
				(v1) --  (f1);
				(v1) --  (f2);
			};
		\end{feynman}
	\end{tikzpicture}
	\caption{Momentum representation of the $\phi^3$ branch process}
	\label{fig:rqft_interactions_momentum-of-phi-cubed-branch}
\end{figure}

We see that conservation of momentum is enforced by the delta function ($p = q + q^\prime$), but we also know that the particles are all necessarily on shell such that
\begin{align*}
	p^2 = q^2 &= (q^\prime)^2 = m^2&
	(p-q)^2 &= m^2&
	2p\vdot q &= m^2.
\end{align*}
Doing the standard trick with rest frames,
\begin{align*}
	p &= (m, \vec{0}) & \vec{q} &= (E, \vec{q}) & \implies \quad 2mE&= m^2
\end{align*}
but
\begin{align*}
	E^2 &= m^2 + \vec{q}^2 \geq m^2
\end{align*}
so for consistency this process \emph{must} vanish.

\subsection{Tree Level}
At $\mathcal{O}(g^2)$ in the tree level, we gain another incoming particle, such that $\ket{i} = \ket{p, p^\prime}$, and a virtual propagating particle with momentum $k$.
The S-channel transition is the matrix element
\begin{align*}
	\mel{f}{S_1^{(2)}}{i} &= -\frac{g^2}{(3!)^2} \int \dd[4]{x} \int \dd[4]{y} \mel{q, q^\prime}{\phi^-(y)^2[3^2\contraction{}{\phi}{(y)}{\phi}\phi(y)\phi(x)]\phi(x)^2}{p, p^\prime}\\
	&= -g^2 \int \dd[4]{x} \int \dd[4]{y} e^{i(q+q^\prime)\vdot{y}}e^{i(p+p^\prime)\vdot{x}} \frac{i}{(2\pi)^4}\int\dd[4]{k} e^{-k\vdot(y-x)}\frac{1}{k^2 + m^2 + i\epsilon}\\
	&= -ig^2\int \frac{\dd[4]{k}}{(2\pi)^4} (2\pi)^8 \delta^4(q + q^\prime - k) \delta^4(k-p-p^\prime)\frac{1}{k^2 + m^2 + i\epsilon}\\
	&= -ig^2 (2\pi)^4 \frac{1}{(p+p^\prime)^2 - m^2 \cancel{+i\epsilon}} \delta^4(q+q^\prime - p - p^\prime)
\end{align*}
in which we can drop the $i\epsilon$ correction, and we can name
\begin{align*}
	\mathcal{M}_s &= -ig^2 (2\pi)^4 \frac{1}{(p+p^\prime)^2 - m^2}
\end{align*}
the Feynman Amplitude for the S-channel.
The delta function in this case enforces that the two incoming particles' momentum is maintained by the two outgoing, and the propagator has momentum $k = p+ p^\prime = q + q^\prime$.

Likewise, the T-channel has matrix element
\begin{align*}
	\mel{f}{S_t}{i} &= -ig^2(2\pi)^4\frac{1}{(p-q)^2 - m^2} \delta^4(q+q^\prime - p - p^\prime)
\end{align*}
and the U-channel has
\begin{align*}
	\mel{f}{S_u}{i} &= -ig^2(2\pi)^4\frac{1}{(p-q^\prime)^2 - m^2} \delta^4(q+q^\prime - p - p^\prime).
\end{align*}
It is clear that these amplitudes vary by how the momentum flows within the diagram. 
We can thus define three invariant quantities, known as the Mandelstam invariants:
\begin{align*}
	s &= (p+p^\prime)^2 = (q+q^\prime)^2\\
	t &= (p-q)^2 = (p^\prime - q^\prime)^2\\
	u &= (p-q^\prime)^2 = (p^\prime-q) 
\end{align*}
which are such that $s+t+u=4m^2$.
In total,
\begin{align*}
	\mel{f}{S_1^{(2)}}{i} &= \mel{q, q^\prime}{S_s + S_t + S_u}{p, p^\prime}\\
	&= -ig^2(2\pi)^4\qty[\frac{1}{s - m^2} + \frac{1}{t - m^2} + \frac{1}{u - m^2}] \delta^4(q + q^\prime - p - p^\prime).
\end{align*}
\subsection{One Loop Diagram}
At $\mathcal{O}(g^2)$ with two contractions, things become very non-trivial; we have $\ket{i}=\ket{p}$ and $\ket{f}=\ket{q}$ and two internally propagating particles between $x$ and $y$.
The transition element is
\begin{align*}
	\mel{f}{S^{(0)}_{2}}{i} &= - \frac{g^2}{\cancel{(3!)^2}}\int\dd[4]{x} \int\dd[4]{y} \mel{q}{\phi^-(y)\cancel{(3!)^2} \qty[\contraction{}{\phi}{(y)}{\phi}\phi(y)\phi(x)]^2\phi^+(x)}{p}\\
	&= -g^2 \int\dd[4]{x} \int\dd[4]{y} e^{iq\vdot{}y} \qty[\frac{i}{(2\pi)^4}]^2\int \dd[4]{k} \frac{e^{ik\vdot{}(x-y)}}{k^2-m^2+i\epsilon} \int \dd[4]{k^\prime} \frac{e^{ik^\prime\vdot{}(x-y)}}{(k^\prime)^2-m^2+i\epsilon} e^{ip\vdot{x}}\\
	&= g^2 \int \frac{\dd[4]{k}}{(2\pi)^4}\int \frac{\dd[4]{k^\prime}}{(2\pi)^4} (2\pi)^8 \delta^4(q-k-k^\prime)\delta^4(k+k^\prime - p) \frac{1}{k^2-m^2+i\epsilon} \frac{1}{(k^\prime)^2-m^2+i\epsilon}\\
	&= (2\pi)^4 \delta^4(q-p) g^2 \int \frac{\dd[4]{k}}{(2\pi)^4} \frac{1}{k^2-m^2+i\epsilon} \frac{1}{(p-k)^2-m^2+i\epsilon}
\end{align*}
which means there's \emph{unbounded} momentum going around in the loop, since all we require is that $p=q$. 
The divergence is clear if $k^2 \gg m^2$ as this means
\begin{align*}
	\mathcal{M}_\text{loop} &\sim g^2 \int \frac{k^2}{k^4}\dd{k^2}
\end{align*}
which (though strange looking) is \emph{logarithmically divergent}.
This is explored further in \autoref{cha:divergences-i-scalar}, in which the divergence is absorbed into renormalisation of the field.


\end{document}


