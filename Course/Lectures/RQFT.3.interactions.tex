% !TeX root = ../notes.tex

\documentclass[notes.tex]{subfiles}


\begin{document}
\chapter{Interactions and Feynman Diagrams}
\label{chap:rqft_interactions}
Thus far, we've only considered the `free field' Lagrangian, entirely quadratic in the fields. Quantisation gave us a field of independent harmonic oscillators, equivalent to entirely non-interacting particles.

This isn't very interesting.

To make it interesting, we add interactions. For scalar fields, the simplest interactions are in what is referred to as $\phi^3$ theory, which is simply where we add \[\mathcal{L}_I = -\frac{1}{6} g \phi^3\] to the free Lagrangian $\mathcal{L}_0$ to get
\begin{align*}
    \mathcal{L} &\equiv \mathcal{L}_0 + \mathcal{L}_I\\
    &= \half(\partial^\mu\phi)(\partial_\mu\phi) - \half m^2\phi^2 - \frac{1}{6}g\phi^3
\end{align*}
where $g$ is referred to as the `coupling constant'.
\end{document}

