% !TeX root = ./notes.tex

\documentclass[notes]{subfiles}

\renewcommand{\tphi}{\tilde{\phi}}
\renewcommand{\tj}{\tilde{J}}
\renewcommand{\tchi}{\tilde{\chi}}
\renewcommand{\psibar}{\bar{\psi}}
\renewcommand{\etabar}{\bar{\eta}}



\begin{document}
\chapter{Path Integrals for Fermion Fields}
\label{cha:path-integr-ferm}
\section{Fermion Path Integral}
\label{sec:ferm-path-integr}

For the case of fermion fields, we want to define the path integral
following the recipe we used for the scalar field. We will treat the
fields $\psi$ and $\bar\psi$ as Grassmann variables, \ie 
\begin{equation}
  \label{eq:GrassVar}
  \left\{\psi_\alpha(x),\psi_\beta(y)\right\}=0\, .
\end{equation}
In order to have a consistent implementation of the anticommuting
properties of the fermion fields, the functional derivative with
respect to a Grassmann variable must be a Grassmann variable
itself. As a consequence
\begin{equation}
  \label{eq:GrassDer}
  \frac{\delta^2 F}{\delta \psi_\alpha(x) \delta\psi_\beta(y)} = 
  -\frac{\delta^2 F}{\delta \psi _\beta(y) \delta\psi_\alpha(x)}\, ,
\end{equation}
and 
\begin{equation}
  \label{eq:GrassDer2}
    \frac{\delta^2 F}{\delta \psi_\alpha(x) \delta\psi_\alpha(x)} = 0\,.
\end{equation}
In the definition of the generating functional, we introduce
independent sources for $\psi$ and $\bar{\psi}$:
\begin{equation}
  \label{eq:SourceTerm}
  \int d^Dy\, \left[
    \bar{\eta}(y) \psi(y) + \bar{\psi}(y) \eta(y)
    \right]\, ,
\end{equation}
such that
\begin{align}
  \label{eq:GrassDerivOne}
  \frac{\delta}{\delta\eta(x)} &\int d^Dy\, \left[
    \bar{\eta}(y) \psi(y) + \bar{\psi}(y) \eta(y)
    \right] = - \psibar(x)\, \\
  \label{eq:GrassDerivTwo}
  \frac{\delta}{\delta\etabar(x)} &\int d^Dy\, \left[
    \bar{\eta}(y) \psi(y) + \bar{\psi}(y) \eta(y)
    \right] = \psi(x)\, .
\end{align}

\paragraph{Free theory}

The action for the free Dirac field is
\begin{equation}
  \label{eq:FreeDiracAction}
  S_0\left[\psi,\psibar\right] = 
  \int d^Dx\, \psibar(x) \left( i \slashed{\partial} -m \right)
  \psi(x)\, .
\end{equation}
 Using the rules above for the functional derivative, we can find the
 classical equation of motion, \ie Dirac's equation
 \begin{align}
   \frac{\delta}{\delta \psibar(x)} S_0\left[\psi,\psibar \right] = 0
   \quad \Longrightarrow \quad
   \left(i \slashed{\partial} - m\right)\psi(x) =0\, .
 \end{align}
By analogy with the scalar case, we can write the generating
functional for the free theory:
\begin{align}
  Z_0\left[\etabar, \eta\right] 
  &= \int \mathcal{D}\psi\, \mathcal{D}\psibar\, 
    \exp\left\{
    i \left( S_0[\psi,\psibar] + \etabar \cdot \psi +
    \psibar \cdot \eta\right)
    \right\} \\
  &= \exp \left[
    - \int d^Dx\, d^Dy\, \etabar(x) S(x-y) \eta(y)
    \right]\, .
\end{align}
The Feynman propagator for the Dirac field is
\begin{equation}
  \label{eq:DiracFeynProp}
  S(x-y) = \int_p e^{-i p\cdot(x-y)}\,
  \frac{i (\slashed{p}+m)}{p^2-m^2+i\epsilon}\, .
\end{equation}
Note that, just like in the case of the scalar field, the propagator
is the inverse of the quadratic term in the action. The propagator is
a $4\times 4$ matrix in spin space, which we can write explicitely:
\begin{align}
  \bra{0} & T \psi_\alpha(x) \psibar_\beta(y) |0 \rangle 
  = \nonumber \\ 
  \label{eq:DiracFeynPropAllIndices}
  &= S_{\alpha\beta}(x-y) = 
  \int_pe^{i p\cdot(x-y)}\,
  \frac{i \left(p_\mu \left(\gamma^\mu\right)_{\alpha\beta}+m\,
    \delta_{\alpha\beta}\right)}
  {p^2-m^2+i\epsilon}\, .
\end{align}
Because of the linear term in $p$ in the propagator the fermionic
propagator is not symmetric in its arguments, and will be denoted with
an arrow pointing from one end to the other:
  \begin{equation}
    \label{eq:SPropXYFeynDiag}
    S(x-y) = 
    \begin{tikzpicture}[baseline={([yshift=1.4ex]current bounding box.center)}]
      \begin{feynman}[inline=(a)]
        \vertex (a);
        \vertex (b);
        \diagram {
          a --[fermion] b,
        };
        \vertex [below=0.2em of a] {\(_{x}\)};  
        \vertex [below=0.2em of b] {\(_{y}\)};  
      \end{feynman}
    \end{tikzpicture}\, .
  \end{equation}

Correlators of fermion fields are computed by taking derivatives with
respect to the source fields
\begin{align}
  \bra{ 0}& T \psi_{\alpha_1}(x_1) \ldots \psibar_{\beta_1}(y_1)
  \ldots |0\rangle_0  = \nonumber \\
   &=\frac{1}{i} \frac{\delta}{\delta\etabar_{\alpha_1}(x_1)} \ldots
  i \frac{\delta}{\delta\eta_{\beta_1}(y_1)}\, 
  \left. Z_0\left[\eta, \etabar\right] \right|_{\eta=\etabar=0}\, .
\end{align}

\paragraph{Interacting theory}

If the interactions are specified by a potential
$V\left(\psi,\psibar\right)$, the generating functional is defined as 
\begin{equation}
  \label{eq:InterGenFunc}
  Z\left[\eta,\etabar\right] \propto
  \exp\left[
    i \int d^Dx\, V\left( i \frac{\delta}{\delta\eta(x)}, 
    \frac{1}{i} \frac{\delta}{\delta\etabar(x)}\right)
  \right] \, 
  Z_0\left[\eta, \etabar \right]\, , 
\end{equation}
and the normalization is fixed by requiring
\begin{equation}
  \label{eq:ZNorm}
  Z[0,0] = 1\, .
\end{equation}
A double expansion in powers of the interaction, and in powers of the
number of propagators defines the interacting path integral, as we did
for the case of scalars. 

\begin{subappendices}
  
\section{Differentiation in Grassmann variables}
\label{sec:diff-grassm-vari}

\paragraph{Grassmann algebra}

A Grassmann algebra $\mathcal{A}$, over $\mathbb{R}$ or $\mathbb{C}$,
is constructed from a set of generators $\theta_i$ satisfying
\begin{equation}
  \label{eq:AntiCommGen}
  \theta_i \theta_j + \theta_j \theta_i = 0\, .
\end{equation}
Note that 
\begin{enumerate}
\item all elements are first degree polynomials in each generator; 
\item if the number of generators is finite and equal to $n$, the
  algebra is vector space of dimension $2^n$.
\end{enumerate}

\paragraph{Grassmannian parity}

Parity is defined as an automorphism on $\mathcal{A}$ is defined by
\begin{equation}
  \label{eq:GrassParity}
  P(\theta_i) = - \theta_i\, .
\end{equation}
The action of $P$ on a monomial is
\begin{equation}
  \label{eq:GrassParMon}
  P(\theta_{i_1} \ldots \theta_{i_p}) = (-)^p \theta_{i_1} \ldots
  \theta_{i_p}\, .
\end{equation}
The reflection defines two eigenspaces containing the even and odd
elements:
\begin{equation}
  \label{eq:ParitySubspaces}
  P(\mathcal{A}^\pm) = \pm \mathcal{A}^\pm\, .
\end{equation}

\paragraph{Grassmann differentiation}

Differentiation is defined as a linear mapping
\begin{equation}
  \label{eq:GrassDiff}
  D: \mathcal{A} \to \mathcal{A}\, ,
\end{equation}
which satisfies
\begin{equation}
  \label{eq:GrassDiffProp}
  D(A_1 A_2) = P(A_1) D(A_2) + D(A_1) A_2\, ,
\end{equation}
which guarantees that
\begin{equation}
  \label{eq:GrassDiffParity}
  D P + P D = 0\, .
\end{equation}
Note that the image of $\mathcal{A}^\pm$ belongs to
$\mathcal{A}^\mp$, \ie derivation changes the parity of product of
Grassmann variables. 

We can introduce the nilpotent differential operators
$\partial/\partial\theta_i$ by
\begin{equation}
  \label{eq:DiffOpsBasis}
  \frac{\partial}{\partial \theta_i} \theta_j = \delta_{ij}\, . 
\end{equation}
The differential operators and the generators can be considered as
operators acting on the elements of $\mathcal{A}$ from the left. They
satisfy the anticommutation relations:
\begin{align}
  \label{eq:AntiComm1}
  \theta_i \theta_j + \theta_j \theta_i &= 0\, ,\\
  \label{eq:AntiComm2}
  \frac{\partial}{\partial\theta_i} \frac{\partial}{\partial\theta_j}
  + \frac{\partial}{\partial\theta_j}
  \frac{\partial}{\partial\theta_i} &= 0\, , \\
  \label{eq:AntiComm3}
  \theta_i \frac{\partial}{\partial\theta_j} +
  \frac{\partial}{\partial\theta_j} \theta_i &= 0\, .
\end{align}

\paragraph{Chain rule}

If $\sigma(\theta)\in\mathcal{A}^-$, $x(\theta)\in \mathcal{A}^+$,
then
\begin{equation}
  \label{eq:GrassChainRule}
  \frac{\partial}{\partial\theta} f(\sigma,x) = 
  \frac{\partial\sigma}{\partial\theta} \frac{\partial
    f}{\partial\sigma} 
  + \frac{\partial x}{\partial\theta} \frac{\partial f}{\partial x}\, .
\end{equation}

\section{Integration in Grassmann variables}
\label{sec:integr-grassm-vari}

To a given differential operator $D$ we associate an integral
operator $I$. The idea is to generalise the concept of \emph{definite}
integral to the case of Grassmann variables. $I$ is defined by
requiring a number of properties that are \emph{expected} to hold for
an integral. 
\begin{enumerate}
\item $I$ is linear
  \begin{equation}
    \label{eq:LinearGrassInt}
    I(\lambda_1 A_1 + \lambda_2 A_2) = 
    \lambda_1 I(A_1) + \lambda_2 I(A_2)\, ;
  \end{equation}
\item $I D = D I = 0\, ;$
\item $D(A) = 0 ~~\Longrightarrow~~ I(BA) = I(B) A\, ;$
\item $P I + I P = 0$.
\end{enumerate}
Note that a nilpotent differentiation operator $D$ satisfies all these
conditions. We shall therefore define the integration operation to be
identical to differentiation:
\begin{equation}
  \label{eq:GrassIntDef}
  \int d\theta_i A \equiv \frac{\partial}{\partial\theta_i} A\, .
\end{equation}

Show that 
\begin{equation}
  \label{eq:GrassJacob}
  \int d\theta f(\theta) = a^{-1} \int d\theta' f(a \theta' + b)\, .
\end{equation}
Note that the Jacobian for this change of variables is $a^{-1}$, \ie
the inverse of the usual Jacobian for commuting variables. You can
prove the generic result
\begin{equation}
  \label{eq:GrassJacobOne}
  \int d\theta_1 \ldots d\theta_n = 
  \int d\theta'_1 \ldots d\theta'_n J(\theta')\, , 
\end{equation}
where 
\begin{equation}
  \label{eq:GrassJacobTwo}
  J^{-1} = \det \frac{\partial \theta_i}{\partial \theta'_j}\, .
\end{equation}

\end{subappendices}

\end{document}