% !TeX root = ../notes.tex

\documentclass[notes.tex]{subfiles}


\begin{document}
\chapter{Canonical Quantisation}
\label{chap:rqft_canonical-quantisation}
\section{Quantum Mechanics}
Consider a point particle with generalised coordinates $q_i$, Lagrangian $L(\{q_i\}, \{\dot{q}_i\})$ and canonical momenta
\begin{align*}
    p_i &=\pdv{L}{q_i}\intertext{such that}
    H &= \sum p_i \dot{q}_i - L
\end{align*}
which define the mechanics of the system. 

The Schr\"odinger picture then views these $\{\hat{q}_i\}$, $\{\hat{p}_i\}$ and $\hat{H}$ as hermitian ($\hat{A}^\dagger = \hat{A}$) on a Hilbert space constructed of the vectors $\ket{\psi, t}$\footnote{Sheer laziness means that these operators can be written without any indication that they are operators and it's usually clear from context what the identity of the thing is.}. This gives us typical quantum mechanical expressions such as
\begin{align*}
    E &= \expval{\hat{H}}{\psi, t}
\end{align*}
and all of the other methods thereby implied.

We now impose the canonical commutation relations
\begin{align*}
    \commutator{\hat{q}_i}{\hat{p}_i} &= i\delta_{ij} & \commutator{\hat{q}_i}{\hat{q}_j} = \commutator{\hat{p}_i}{\hat{p}_j} &= 0
\end{align*}
and note that the time evolution of these states is given by the Schr\"odinger equation
\begin{align*}
    i\dv{t}\ket{\psi, t} &= \hat{H}\ket{\psi, t}\\
    \implies \quad \ket{\psi, t} &= e^{-i\hat{H}t}\ket\psi
\end{align*}
which emphasises that this has \emph{time dependent} wavefunctions and \emph{time independent} operators.

There's an alternative but entirely equivalent viewpoint on time evolution, which is the Heisenberg Picture. This moves the time dependency from the states to the operators, such that
\begin{align*}
    \underbrace{\hat{A}(t)}_{\text{Heisenberg Picture}} &=\underbrace{ e^{i\hat{H}t} \hat{A} e^{-i\hat{H}t}}_{\text{Schr\"odinger Picture}}
\end{align*}
and it's trivial to show that
\begin{align*}
    \expval{\hat{A}} &\equiv \expval{\hat{A}}{\psi, t} \\
    &= \expval{\hat{A}(t)}{\psi}
\end{align*}
which is good, because it means that the physics defined by the observables is unaffected by the way we formulate the mathematics.
We also know that the commutation relations are unchanged such that
\begin{align*}
    \commutator{\hat{q}_i(t)}{\hat{p}_i(t)} &= i\delta_{ij}
\end{align*}
for any arbitrary time $t$.
The time dependence of the operators is given by the Heisenberg time evolution equation
\begin{align*}
    \dv{t}\hat{A}(t) &= i\hat{H} e^{i\hat{H}}\hat{A} e^{-i\hat{H}} - e^{i\hat{H}}\hat{A} i\hat{H} e^{-i\hat{H}} \\
    &= i\commutator{\hat{H}}{\hat{A}(t)}
\end{align*}
which also comes from the Poisson Bracket definition of a symplectic transform from Hamiltonian Dynamics. 

\begin{example}
    Consider a quantum particle moving in a potential $V(q)$ such that
    \begin{align*}
        \hat{L} &= \frac{1}{2}m\dot{\hat{q}}^2 - V(\hat{q})\\
        \hat{H} &= \frac{\hat{p}^2}{2m} + V(\hat{q}).
    \end{align*}
    We can derive that
    \begin{align*}
        \hat{\dot{q}} &= i\commutator{\hat{H}}{\hat{q}}\\
        &= \frac{i}{2m}\commutator{\hat{p}^2}{\hat{q}}\\
        &= \frac{i}{2m}\qty(\hat{p}\commutator{\hat{p}}{\hat{q}} + \commutator{\hat{p}}{\hat{q}}\hat{p})\\
        &= \frac{-2i^2}{2m}\hat{p} = \frac{\hat{p}}{m}
    \end{align*}
    and
    \begin{align*}
        \hat{\dot{p}} &= i\commutator{\hat{H}}{\hat{p}}\\
        &= i\commutator{V(\hat{q})}{\hat{p}}\\
        &= -\pdv{V}{\hat{q}}
    \end{align*}
    which aligns with the Hamiltonian equations of motion \autoref{eqn:rqft_cft_hamiltonian-eom}.
\end{example}

\section{Quantum Fields}
Having done all of the work with quantum mechanics, we can now interpret $\phi(x)$ and the conjugate momentum $\pi(x)$ as Heisenberg operators, with some adjustments for the continuous nature of the fields. 
The main correction for continuity is that we have to adjust the canonical commutation relation. 
This has a complication that our previous commutation relations were always at \emph{equal time}, which is explicitly not Lorentz invariant, since it treats time differently to the spatial dimensions. However, we plough on regardless and assert that
\begin{align*}
    \commutator{\phi(t, \vec{x})}{\pi(t, \vec{x}^\prime)} &= i \delta^3(\vec{x} - \vec{x}^\prime)\\
    \commutator{\phi(t, \vec{x})}{\phi(t, \vec{x}^\prime)} &= 0\\
    \commutator{\pi(t, \vec{x})}{\pi(t, \vec{x}^\prime)} &= 0
\end{align*}
which are called the Equal Time Commutation Relations (ETCRs). 
This is inherently a multiparticle formulation, which is nice, and also probably Lorentz invariant as $t$ and $\vec{x}$ are both treated as pure parameters to the field $\phi$.

\subsection{Equations of Motion for the Field}
As we've established in the previous section, we can use Heisenberg's equation of motion to derive the kinematics of the system.
For example, 
\begin{align*}
    \dot\psi(t, \vec{x}) &= i \commutator{H}{\psi{t, \vec{x}}}\\
    &= i\int \commutator{\mathcal{H}(t, \vec{x^\prime})}{\phi(t, \vec{x}^\prime)} \dd[3]{\vec{x}}
\end{align*}
but, using the argument seen previously, we know that
\begin{align*}
    \commutator{\mathcal{H}(t, \vec{x}^\prime)}{\phi(t, \vec{x})} &= -i\pi(t, \vec{x}^\prime)\, \delta^3(\vec{x} - \vec{x}^\prime)
\end{align*}
using the equal time commutation relations in place of the canonical. 
Thus
\begin{align*}
    \dot{\phi}(t, \vec{x}) &= \int \pi(t, \vec{x}^\prime)\, \delta^3(\vec{x} - \vec{x}^\prime) \dd[3]\vec{x} \\
    &= \pi(t, \vec{x})
\end{align*}
\end{document}

