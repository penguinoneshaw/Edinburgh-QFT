% !TeX root = ../notes.tex

\documentclass[notes.tex]{subfiles}


\begin{document}
\chapter{Canonical Quantisation}
\label{chap:rqft_canonical-quantisation}
\section{Quantum Mechanics}
Consider a point particle with generalised coordinates $q_i$, Lagrangian $L(\{q_i\}, \{\dot{q}_i\})$ and canonical momenta
\begin{align*}
    p_i &=\pdv{L}{q_i}\intertext{such that}
    H &= \sum p_i \dot{q}_i - L
\end{align*}
which define the mechanics of the system. 

The Schr\"odinger picture then views these $\{\hat{q}_i\}$, $\{\hat{p}_i\}$ and $\hat{H}$ as hermitian ($\hat{A}^\dagger = \hat{A}$) on a Hilbert space constructed of the vectors $\ket{\psi, t}$\footnote{Sheer laziness means that these operators can be written without any indication that they are operators and it's usually clear from context what the identity of the thing is.}. This gives us typical quantum mechanical expressions such as
\begin{align*}
    E &= \expval{\hat{H}}{\psi, t}
\end{align*}
and all of the other methods thereby implied.

We now impose the canonical commutation relations
\begin{align*}
    \commutator{\hat{q}_i}{\hat{p}_i} &= i\delta_{ij} & \commutator{\hat{q}_i}{\hat{q}_j} = \commutator{\hat{p}_i}{\hat{p}_j} &= 0
\end{align*}
and note that the time evolution of these states is given by the Schr\"odinger equation
\begin{align*}
    i\dv{t}\ket{\psi, t} &= \hat{H}\ket{\psi, t}\\
    \implies \quad \ket{\psi, t} &= e^{-i\hat{H}t}\ket\psi
\end{align*}
which emphasises that this has \emph{time dependent} wavefunctions and \emph{time independent} operators.

There's an alternative but entirely equivalent viewpoint on time evolution, which is the Heisenberg Picture. This moves the time dependency from the states to the operators, such that
\begin{align*}
    \underbrace{\hat{A}(t)}_{\text{Heisenberg Picture}} &=\underbrace{ e^{i\hat{H}t} \hat{A} e^{-i\hat{H}t}}_{\text{Schr\"odinger Picture}}
\end{align*}
and it's trivial to show that
\begin{align*}
    \expval{\hat{A}} &\equiv \expval{\hat{A}}{\psi, t} \\
    &= \expval{\hat{A}(t)}{\psi}
\end{align*}
which is good, because it means that the physics defined by the observables is unaffected by the way we formulate the mathematics.
We also know that the commutation relations are unchanged such that
\begin{align*}
    \commutator{\hat{q}_i(t)}{\hat{p}_i(t)} &= i\delta_{ij}
\end{align*}
for any arbitrary time $t$.
The time dependence of the operators is given by the Heisenberg time evolution equation
\begin{align*}
    \dv{t}\hat{A}(t) &= i\hat{H} e^{i\hat{H}}\hat{A} e^{-i\hat{H}} - e^{i\hat{H}}\hat{A} i\hat{H} e^{-i\hat{H}} \\
    &= i\commutator{\hat{H}}{\hat{A}(t)}
\end{align*}
which also comes from the Poisson Bracket definition of a symplectic transform from Hamiltonian Dynamics. 

\begin{example}
    Consider a quantum particle moving in a potential $V(q)$ such that
    \begin{align*}
        \hat{L} &= \half{}m\dot{\hat{q}}^2 - V(\hat{q})\\
        \hat{H} &= \frac{\hat{p}^2}{2m} + V(\hat{q}).
    \end{align*}
    We can derive that
    \begin{align*}
        \hat{\dot{q}} &= i\commutator{\hat{H}}{\hat{q}}\\
        &= \frac{i}{2m}\commutator{\hat{p}^2}{\hat{q}}\\
        &= \frac{i}{2m}\qty(\hat{p}\commutator{\hat{p}}{\hat{q}} + \commutator{\hat{p}}{\hat{q}}\hat{p})\\
        &= \frac{-2i^2}{2m}\hat{p} = \frac{\hat{p}}{m}
    \end{align*}
    and
    \begin{align*}
        \hat{\dot{p}} &= i\commutator{\hat{H}}{\hat{p}}\\
        &= i\commutator{V(\hat{q})}{\hat{p}}\\
        &= -\pdv{V}{\hat{q}}
    \end{align*}
    which aligns with the Hamiltonian equations of motion \autoref{eqn:rqft_cft_hamiltonian-eom}.
\end{example}

\section{Quantum Fields}
Having done all of the work with quantum mechanics, we can now interpret $\phi(x)$ and the conjugate momentum $\pi(x)$ as Heisenberg operators, with some adjustments for the continuous nature of the fields. 
The main correction for continuity is that we have to adjust the canonical commutation relation. 
This has a complication that our previous commutation relations were always at \emph{equal time}, which is explicitly not Lorentz covariant, since it treats time differently to the spatial dimensions. However, we plough on regardless and assert that
\begin{align*}
    \commutator{\phi(t, \vec{x})}{\pi(t, \vec{x}^\prime)} &= i \delta^3(\vec{x} - \vec{x}^\prime)\\
    \commutator{\phi(t, \vec{x})}{\phi(t, \vec{x}^\prime)} &= 0\\
    \commutator{\pi(t, \vec{x})}{\pi(t, \vec{x}^\prime)} &= 0
\end{align*}
which are called the Equal Time Commutation Relations (ETCRs). 
This is inherently a multiparticle formulation, which is nice, and also probably Lorentz invariant as $t$ and $\vec{x}$ are both treated as pure parameters to the field $\phi$.

\subsection{Equations of Motion for the Field}
As we've established in the previous section, we can use Heisenberg's equation of motion to derive the kinematics of the system.
For example, 
\begin{align*}
    \dot\psi(t, \vec{x}) &= i \commutator{H}{\psi{t, \vec{x}}}\\
    &= i\int \commutator{\mathcal{H}(t, \vec{x^\prime})}{\phi(t, \vec{x}^\prime)} \dd[3]{\vec{x}}
\end{align*}
but, using the argument seen previously, we know that
\begin{align*}
    \commutator{\mathcal{H}(t, \vec{x}^\prime)}{\phi(t, \vec{x})} &= -i\pi(t, \vec{x}^\prime)\, \delta^3(\vec{x} - \vec{x}^\prime)
\end{align*}
using the equal time commutation relations in place of the canonical. 
Thus
\begin{align*}
    \dot{\phi}(t, \vec{x}) &= \int \pi(t, \vec{x}^\prime)\, \delta^3(\vec{x} - \vec{x}^\prime) \dd[3]\vec{x} \\
    &= \pi(t, \vec{x})
\end{align*}
and similarly for the conjugate momentum
\begin{align*}
    \commutator{\mathcal{H}(t, \vec{x}^\prime)}{\pi(t, \vec{x})} &= \commutator{\cancel{\half{}\pi^2(x^\prime)}+\half{}(\grad^\prime\phi(x^\prime))^2 + \half{}m^2\phi^2(x^\prime)}{\pi(x)}\\
    &= \half{}\qty(\grad^\prime\phi(x^\prime)\vdot\grad^\prime\commutator{\phi(x^\prime)}{\pi(x)} + \grad^\prime\commutator{\phi(x^\prime)}{\pi(x)}\vdot \grad^\prime\phi(x^\prime))\\
    &\qquad + im^2\phi(x^\prime)\delta^3(\vec{x} - \vec{x}^\prime)\\
    &= i\qty(\grad^\prime\phi(x^\prime)\vdot\grad^\prime + m^2 \phi)\delta^3(\vec{x} - \vec{x}^\prime)
\end{align*}
where we have used $\pi(x)$ indpendence from $x^\prime$ to take the derivative out of the commutator. 
We then use integration by parts\footnote{\begin{align*}
    \int \grad^\prime\phi(x^\prime)\vdot\grad^\prime \delta^3(\vec{x} - \vec{x}^\prime) \dd[3]{\vec{x}^\prime} &= \qty[\grad^\prime\phi(x^\prime) \delta^3(\vec{x} - \vec{x}^\prime)]_{-\infty}^\infty - \int (\laplacian)^\prime \phi(x^\prime) \delta^3(\vec{x} - \vec{x}^\prime) \dd[3]{\vec{x}}
\end{align*}} to do
\begin{align*}
    \dot{\pi}(t, \vec{x}) &= i\commutator{H}{\pi(t, \vec{x})}\\
    &= i \int \commutator{\mathcal{H}(t, \vec{x}^\prime)}{\pi(t, \vec{x})}\dd[3]{\vec{x}^\prime}\\
    &= - \int \qty(\grad^\prime\phi(x^\prime)\vdot\grad^\prime + m^2 \phi)\delta^3(\vec{x} - \vec{x}^\prime)  \dd[3]{\vec{x}^\prime}\\
    \ddot{\phi} = \dot{\pi} &= (\laplacian-m^2)\phi(t, \vec{x})
\end{align*}
and we notice that though the equal time commutation relations are not covariant, we still get the covariant equations of motion.

\section{Quantum Harmonic Oscillator}
A quantum oscillator is defined by the Hamiltonian
\begin{align*}
    \hat{H} &= \frac{\hat{p}^2}{2m} + \half{} m \omega^2 \hat{q}^2
\end{align*}
and we can, from this definition, define two extra operators, $\hat{a}$ and $\hat{a}^\dagger$, such that
\begin{align*}
    \hat{q} &= \frac{1}{\sqrt{2m\omega}}(\hat{a} + \hat{a}^\dagger) & \hat{p} &= -\sqrt{\frac{m\omega}{2}} (\hat{a} - \hat{a}^\dagger)
\end{align*}
and which are consistent with the canonical commutation relations in that
    $\commutator{\hat{a}}{\hat{a}^\dagger} = 1$
and
\begin{align*}
    H &= \omega\qty(\hat{a}^\dagger\hat{a} + \half{}).
\end{align*}
We can thus define the (hermitian) number operator which defines the spectrum of $H$, $\hat{N} = \hat{a}^\dagger\hat{a}$, for which the eigenvalues are the energy of the system.
We can thus define eigenvectors for this operator as 
\begin{align*}
    \hat{N}\ket{n} &= n\ket{n} & \braket{n}{n} &= 1
\end{align*}
for which $n>0$ for all $n$.

We can show relatively easily that $\hat{a}$ and $\hat{a}^\dagger$ are the ladder operators such that (for $n\geq 0$)
\begin{align*}
    \hat{a}^\dagger \ket{n} &= \sqrt{n+1}\ket{n+1} & \hat{a}\ket{n} &= \sqrt{n}\ket{n-1}
\end{align*}
and thus that all states of the oscillator with energy eigenvalue $(n+\half{})\omega$ can be written as 
\begin{align*}
    \ket{n} &= \frac{(\hat{a}^\dagger)^n}{\sqrt{n}}\ket{0}
\end{align*}
where $\ket{0}$ is the ground of \emph{vaccuum} state.

\section{Two Interpretations}
When formulating quantum mechanics, there are two competing interpretations known as the first and second quantisations.

The first quantisation is the Schr\"odinger equation based wavefunction interpretation. This interprets the harmonic oscillator in terms of its classical equivalent, just with discrete energy levels which are evenly spaced. 
This has ``ground states'' 
\begin{align*}
    \psi_0(x) &= \sqrt{\frac{m\omega}{2\pi}} e^{-\half{}m\omega^2x^2}
\end{align*}
which are then excited to
\begin{align*}
    \psi_n(x) &= \frac{1}{\sqrt{n!}} \qty(\frac{-i}{\sqrt{2m\omega}} + i\sqrt{\frac{m\omega}{2}}x)^n \psi_n(x).
\end{align*}
This is inherently a single particle interpretation.

Heisenberg based the second quantisation on the idea that a state $\ket{n}$ consists of $n$ \emph{identical} quanta, each with energy $\omega$.
We then interpret $\ket{0}$ as the vaccuum state with no quanta at all, and interpret the oscillator as a black box, forgetting $\hat{x}$ and $\hat{p}$.

Transitions which can then happen inside the box are 
\begin{align*}
    \ket{n}&\overbrace{\to\ket{n+1}}^{\text{absorb a quantum}}&\ket{n}&\overbrace{\to\ket{n-1}}^\text{emit a quantum}
\end{align*}
which correspond to the $\hat{a}^\dagger$ (creation) and $\hat{a}$ (annihilation) operators. This is fundamentally a multiparticle interpretation.

\subsection{Heisenberg Picture Ladder Operators}
The ladder operators developed thus far were in the Schr\"odinger picture; we now find them in the Heisenberg (\ie the time dependent equivalent). 

Using the Heisenberg equation, we know that $a(t)$ and $a^\dagger(t)$ should satisfy
\begin{align*}
    \dot{a}(t) &= i\commutator{H}{a(t)} & \dot{a}^\dagger(t) &= i\commutator{H}{a^\dagger(t)}\\
    &= i\commutator{\omega(a^\dagger(t)a(t)) + \half{}}{a(t)} &&= i\commutator{\omega(a^\dagger(t)a(t)) + \half{}}{a^\dagger(t)}\\
    &= \omega i\commutator{a^\dagger(t)a(t)}{a(t)} &&= \omega i\commutator{a^\dagger(t)a(t)}{a^\dagger(t)}\\
    &= \omega i (a^\dagger(t) \commutator{a(t)}{a(t)} + \commutator{a^\dagger(t)}{a(t)}a(t))&&=\omega i (a^\dagger(t) \commutator{a(t)}{a^\dagger(t)} + \commutator{a^\dagger(t)}{a^\dagger(t)}a(t))\\
    &= -i\omega a(t) &&= i\omega a^\dagger(t)
\end{align*}
thus
\begin{align*}
    a(t) &= e^{-\omega t} \hat{a} & a^\dagger(t) &= e^{i\omega t} \hat{a}^\dagger
\end{align*}
which we should've expected, since this is also the result we get from the Schr\"odinger equation.

\section{Mode Expansion of (Quantum) Fields}
Since $\phi(x)$ satisfies the Klein-Gordon equation, it has plane wave solutions $e^{-ipx}$ where $p^2 = m^2$ (i.e.~requiring that the momentum is `on-shell'). 
By Fourier's theorem, we can expand any field in terms of plane waves. 
We do the Fourier expansion in its four-vector basis, and add corrections to require the on-shell condition and positive energy; the coefficients are suggestively labelled but need to be confirmed to be the ladder operators, but nevertheless we see that
\begin{align*}
    \phi(x) &= \int \frac{\dd[4]{p}}{(2\pi)^4} 2\pi\delta(p^2 - m^2)\qty(a(\vec{p})e^{-ip\vdot x} + a^\dagger(\vec{p})e^{ip\vdot x})\Theta(p^0)\\
    &= \int \frac{\dd[3]{\vec{p}}}{(2\pi)^3} \int \dd{p^0}\frac{1}{2p^0}\delta(p^0 - \omega(\vec{p}))\qty(a(\vec{p})e^{-ip\vdot x} + a^\dagger(\vec{p})e^{ip\vdot x})\\
    &= \eval{\int \frac{\dd[3]{\vec{p}}}{(2\pi)^3 2\omega(\vec{p})} \qty(a(\vec{p})e^{-ip\vdot x} + a^\dagger(\vec{p})e^{ip\vdot x}) }_{p_0 = \omega(\vec{p})}
\end{align*}
which has a few interesting features:
\begin{enumerate}
    \item This expansion is \emph{manifestly} covariant;
    \item $p^0 = \omega(\vec{p}) = \sqrt{\vec{p}^2 + m^2}$ is always positive (by design);
    \item Implicitly, we have decided that $e^{-ip\vdot x} = e^{-i\omega(\vec{p})t + i \vec{p}\vdot\vec{p}}$ as $p$ is on-shell;
    \item Since $\phi(x)$ is a Heisenberg operator, $a(\vec{p})$ and $a^\dagger (\vec{p})$ are (time independent) operators;
    \item We need both $a(\vec{p})$ and $a^\dagger(\vec{p})$ to ensure that $\phi(x)$ is hermitian;
    \item The integration measure $\int\dds{p} = \int\frac{\dd[3]{\vec{p}}}{(2\pi)^3 2\omega(\vec{p})}$ which is helpfully Lorentz invariant.
\end{enumerate}

The conjugate field is
\begin{align*}
    \pi(x) = \dot\phi(x) = -i\int\dds{p}\omega(\vec{p})\qty(a(\vec{p})e^{-ip\vdot x} - a^\dagger(\vec{p})e^{ip\vdot x}).
\end{align*}

In order to find expressions for $a(\vec{p})$ and $a^\dagger(\vec{p})$, we invert the representations above
\begin{align*}
    \int e^{ip^\prime\vdot x}\phi(x)\dd[3]{\vec{x}} &= \int \dds{p}\int \dd[3]{\vec{x}}\qty(a(\vec{p})e^{-i(p^\prime-p)\vdot x} + a^\dagger(\vec{p})e^{i(p^\prime+p)\vdot x})\\
    &= \int \frac{\dd[3]{\vec{p}}}{2\omega(\vec{p})}\qty[a(\vec{p}) \delta^3(\vec{p} - \vec{p}^\prime) e^{i(\omega(\vec{p}^\prime) - \omega(\vec{p}))t} + a^\dagger(\vec{p}) \delta^3(\vec{p} + \vec{p}^\prime) e^{i(\omega(\vec{p}^\prime) + \omega(\vec{p}))t}]\\
    &= \frac{1}{2\omega(\vec{p}^\prime)}\qty[a(\vec{p}^\prime) + a^\dagger(-\vec{p})e^{2i\omega(\vec{p}^\prime)t}]
\end{align*} 
and similarly
\begin{align*}
    \int e^{ip^\prime\vdot{x}}\pi(x)\dd[3]\vec{x} &= -\frac{i}{2}\qty[a(\vec{p}^\prime) - a^\dagger(-\vec{p}^\prime)e^{i\omega(\vec{p})t}]
\end{align*}
where it should be noted that these are not the full fourier transform representations, as the time coordinate has not been transformed and that $\omega(\vec{p}) = \omega(-\vec{p})$.
Therefore we find that
\begin{align*}
a(\vec{p}) &= \int\dd[3]{\vec{x}} e^{ip\vdot{x}} (\omega(\vec{p})\phi(t, \vec{x}) + i \pi(t, \vec{x}))\\
a^\dagger(\vec{p}) &= \int\dd[3]{\vec{x}} e^{-ip\vdot{x}} (\omega(\vec{p})\phi(t, \vec{x}) - i \pi(t, \vec{x})) 
\end{align*}
and it is trivial to show that
\begin{align*}
    \commutator{a(\vec{p})}{a^\dagger(\vec{p}^\prime)} &= 2\omega(\vec{p})(2\pi)^3 \delta^3(\vec{p}-\vec{p}^\prime)\\
    &= \deltaslash^3(\vec{p} - \vec{p}^\prime)
\end{align*}
where we've invented a notation for the normalised Dirac delta for the covariant measure. 
Similarly, we can show that \begin{align*}
    \commutator{a(\vec{p})}{a(\vec{p}^\prime)} &= \commutator{a^\dagger(\vec{p})}{a^\dagger(\vec{p}^\prime)}=0
\end{align*}
which confirms that these operators conform to the same commutation relations as the regular quantum harmonic oscillator.

Using a few tricks, including that it must be time independent and that we can split the density linearly, we can derive that 
\begin{align*}
    H &= \half{}\int\!\! \omega(\vec{p})[a(\vec{p})a^\dagger(\vec{p}) + a^\dagger(\vec{p})a(\vec{p})]\,\dds{p}
\end{align*}
which makes $ \omega(\vec{p})$ the energy of the normal mode $p$ to the field.
We can also deduce commutators (\ie the time dependence) as
\begin{align*}
    \commutator{H}{a^\dagger(\vec{p})} &= \omega(\vec{p}) a^\dagger(\vec{p})\\
    \commutator{H}{a(\vec{p})} &= -\omega(\vec{p}) a(\vec{p})
\end{align*}
so we have a quantum mechanical simple harmonic oscillator with energy $\omega(\vec{p}) = \sqrt{\vec{p}^2 + m^2}$ for every Fourier mode $p$.

\section{Particle Interpretation}
Consider some energy eigenstate $\ket{E}$. 

The state $a^\dagger(\vec{p})\ket{E}$ has energy $E+ \omega(\vec{p})$ which we get from the application of the commutator
\begin{align*}
    H a^\dagger(\vec{p}) \ket{E} &= \commutator{H}{a^\dagger(\vec{p})}\ket{E} + a^\dagger(\vec{p})H \ket{E}\\
    &= ( \omega(\vec{p}) + E)\, a^\dagger(\vec{p}) \ket{E}
\end{align*}
and similarly $a(\vec{p})\ket{E}$ has energy $E- \omega(\vec{p})$ so as with the single-particle version, we have `creation' and `annihilation' operators for quanta with $\omega(\vec{p}) = \sqrt{m^2 + \vec{p}^2}$. 

Continuing on that line of reasoning, we also know that the Hamiltonian is positive definite, so there must be a state of lowest energy, the vaccuum state $\ket{0}$ for which $$a(\vec{p})\ket{0} = 0 \quad\forall\vec{p}.$$
There is then the Fock space --- the full space of states which may be constructed through applying $a^\dagger(\vec{p})$ to $\ket{0}$ with the necessary $\vec{p}$ states. This creates a state space in which all states have the form 
\begin{align*}
    a^\dagger(\vec{p}_1)\cdots a^\dagger(\vec{p}_n)\ket{0}
\end{align*}
which are all energy eigenstates with eigenvalues $\omega(\vec{p}_1) + \cdots + \omega(\vec{p}_n)$. We choose to normalize such that $\braket{0}{0} = 1$ and \[\braket{\vec{p}}{\vec{p}^\prime} = \mel{0}{a(\vec{p})\,a^\dagger(\vec{p})}{0} = \mel{0}{\commutator{a(\vec{p})}{a^\dagger(\vec{p})}}{0} = \deltaslash^3(\vec{p} - \vec{p}^\prime)\]

This has the side-effect of ensuring the particles follow Bose-Einstein statistics: all $a^\dagger(\vec{p})$ commute with each other, so particles (quanta) can be arbitrarily swapped, such as in the two particle state 
\begin{align*}
    \ket{\vec{p}; \vec{p}^\prime} &= \ket{\vec{p}}\ket{\vec{p}^\prime}\\
    &= a^\dagger(\vec{p}) a^\dagger(\vec{p}^\prime)\ket{0}.
\end{align*}

\section{Ground State Energy}
What is the energy of the vaccuum state? We start from
\begin{align*}
    H &= \int\dds{p} \omega(\vec{p})\qty(a^\dagger(\vec{p}) a(\vec{p}) + \half{}\commutator{a(\vec{p})}{a^\dagger(\vec{p})})\\
    &= \int \dds{p}\omega(\vec{p})\qty(N(\vec{p}) + \omega(\vec{p})(2\pi)^3 \delta^3(0))
\end{align*}
where $N(\vec{p})=a^\dagger(\vec{p}) a(\vec{p})$ is the number density operator. 
Clearly by the way it's defined, $\mel{0}{N(\vec{p})}{0} = 0$, so 
\begin{align*}
    \mel{0}{H}{0} &= \half{}\int\dd[3]\vec{p} \omega(\vec{p}) \delta^3(0)
\end{align*}
which is infinite because of the $\delta^3(0)$.

Each point oscillates at the field's zero point energy on the infinite lattice. We ignore this and hope for the best, since infinite energy is usually a bit of a problem. Generally, we \emph{measure} the energy differences, so the zero point energy issue is academic.

\subsection{Normal Ordering}
The formal system for eliminating the zero point energy is to use `normal ordering', in which we reorder the operators such that all $a(\vec{p})$ are to the right of all $a^\dagger(\vec{p})$. 

For example, if we write 
\begin{align*}
    \phi(x) &= \phi^+(x) + \phi^-(x)
\end{align*}
where
\begin{align*}
    \phi^+(x) &= \int \dds{p} a(\vec{p})e^{-ip\vdot{x}}&
    \phi^-(x) &= \int \dds{p} a(\vec{p}^\dagger)e^{ip\vdot{x}}&
    \phi^-(x) &= \qty(\phi^+)^\dagger
\end{align*}
we can consider
\begin{align*}
    \phi(x)\phi(y) &= \phi^+(x)\phi^+(y) + \phi^+(x)\phi^-(y) + \phi^-(x)\phi^+(y) + \phi^-(x)\phi^-(y).
\end{align*}
We then define that
\begin{align*}
    \normalorder{\phi(x)\phi(y)} &= \phi^+(x)\phi^+(y) + \underline{\phi^-(y)\phi^+(x)} + \phi^-(x)\phi^+(y) + \phi^-(x)\phi^-(y)
\end{align*}
so clearly
\begin{align*}
    \phi^+\ket{0} &= 0 & \phi^-\ket{0} &= 0 & \mel{0}{\normalorder{\phi(x)\phi(y)}}{0} &= 0
\end{align*} 
which is true for any natural monomial.
This means that subtracting the zero point energy is equivalent to normally ordering the Hamiltonian
\begin{align*}
    H \to \normalorder{H} = H - \mel{0}{H}{0}.
\end{align*}

\subsection{Ground State Momentum}
The classical momentum operator is defined as
\begin{align*}
    \vec{P} &= \int \dd[3]\vec{x} \pi(x)\grad{\phi}(x)\\
    &= \half\int\dds{p} \vec{p} \qty[a(\vec{p})a^\dagger(\vec{p}) + a^\dagger(\vec{p}) a(\vec{p})]\\
    &= \int \vec{p}\qty[N(\vec{p}) + \omega(\vec{p})(2\pi)^3 \delta^3(0)] \,\dds{p} 
\end{align*}
for which the infinite term is formally zero, as the integrand is symmetric under parity. 
Thus
\begin{align*}
    \mel{0}{\vec{p}}{0} &= \mel{0}{\normalorder{\vec{p}}}{0} = 0
\end{align*}
\ie{}the vaccuum has no momentum and the state $a^\dagger(\vec{p}_1)\cdots a^\dagger(\vec{p}_n)\ket{0}$ has momentum $p = \sum\qty(\omega(\vec{p}_i), \vec{p}_i)$.

\section{Covariant Commutators}
So far, we have considered the Equal Time Commutation Relations. What we need to know is whether these give Lorentz invariant quantum fields, which it transpires they do.

The \emph{un}equal time commutators,
\begin{align*}
    \commutator{\phi(x)}{\phi(y)} &= \commutator{\phi^+(x)}{\phi^-(y)} + \commutator{\phi^-(x)}{\phi^+(y)}
\end{align*}
where
\begin{align*}
    \commutator{\phi^+(x)}{\phi^-(y)} &= \int \dds{p} \int \dds{p^\prime} \commutator{a(p)}{a^\dagger(p)}e^{-ip\vdot x + i p^\prime \vdot y}\\
    &= \int\dds{p}e^{-ip\vdot(x - y)}\\
    \commutator{\phi^-(x)}{\phi^+(y)} &= \int \dds{p} \int \dds{p^\prime} \commutator{a^\dagger(p)}{a(p)}e^{+ip\vdot x - i p^\prime \vdot y}\\
    &= -\int\dds{p}e^{ip\vdot(x - y)}
\end{align*}
lead us to define a new set of symbols $\Delta^\pm(x)$ and $\Delta(x)$ such that
\begin{align}
    i\Delta^+(x)&\equiv\int\dds{p}e^{-ipx} & i\Delta^-(x)&\equiv -\int\dds{p}e^{ipx} & \Delta(x) &\equiv \Delta^+(x) + \Delta^-(x).\label{eqn:rqft_canonical-quantisation_feynprops}
\end{align}
This gives us
\begin{align*}
    \commutator{\phi^+(x)}{\phi^-(y)} &= i\Delta^+(x-y) & \commutator{\phi^-(x)}{\phi^+(y)} &= i\Delta^-(x-y)
\end{align*}
so 
\begin{align*}
    \commutator{\phi(x)}{\phi(y)} &= i\Delta(x-y)\\
    &= 2\int\sin(p\vdot{x})\dds{p}
\end{align*}
which means it is odd under parity and real! 
Alternatively, 
\begin{align*}
    i\Delta(x) &= \int \frac{\dd[4]{p}}{(2\pi)^4} 2\pi\delta(p^2 - m^2)\epsilon(p^0) e^{-ip\vdot x}
\end{align*}
where $\epsilon(p^0) = \Theta(p^0) - \Theta(-p^0)$\footnote{Where $\Theta(x)$ is the Heaviside Theta function, which is such that $\dv{x}\Theta(x) = \delta(x)$.} means that all of the $\Delta^{(\pm)}(x)$ objects are invariant under proper Lorentz transformations.

\section{Microcausality}
Assume $x$ is spacelike separated from the origin (\ie{$x^2<0$}) such that
\begin{align*}
    \commutator{\phi(x)}{\phi(0)} &= i\Delta(x)
\end{align*}
which allows us to use a proper Lorentz transformation to do $x\mapsto-x$.
However, $\Delta^\pm$ are independently Lorentz covariant such that
\begin{align*}
    \Delta(x) &= \Delta^+(x) + \Delta^-(x)\\
    &= \Delta^+(x) + \Delta^-(-x)\\
    &= \Delta^+(x) - \Delta^-(x)\\
    &= 0
\end{align*}
meaning $\commutator{\phi(x)}{\phi(y)} = 0$ for all spacelike separated points.
If the points commute, this means the fields commute. On the other hand, timelike or lightlike separated points (where $(x-y)^2 \geq 0$) \emph{do not commute}.

The Equal Time commutation relation is a particular example of this, with
\begin{align*}
    \eval{\pdv{x^0}\Delta(x-y)}_{x_0 = y_0} &= i\int \frac{\dd[3]{\vec{p}}}{(2\pi)^3 2 \cancel{\omega(\vec{p})}} i \cancel{\omega(\vec{p})} \qty[e^{ip\vdot(x-y)} - e^{-ip\vdot(x-y)}]\\
    &= -\delta^3(\vec{x} - \vec{y})
\end{align*}
so\begin{align*}
    \commutator{\pi(t, \vec{x})}{\phi(t, \vec{y})} &= -i\delta^3(\vec{x} - \vec{y})
\end{align*}

\subsection{Contour Representation}
It is useful for us to write 
\begin{align*}
    \Delta^\pm &= - \oint_{C_\pm} \frac{\dd[4]{p}}{(2\pi)^4}\frac{e^{-ip\vdot{}x}}{p^2 - m^2}
\end{align*}
where $C_\pm$ are positively oriented circular contours at $p^0=\pm \omega(\vec{p})$ which are the poles of the integrand. This gives the same expressions as were given in \autoref{eqn:rqft_canonical-quantisation_feynprops}, and makes obvious that $\Delta^{(\pm)}(x)$ are Green's functions for the Klein Gordon equation such that
\begin{align*}
    (\partial^2 + m^2)\Delta^{(\pm)}(x) &= \delta^4(x)
\end{align*}
which is amazingly useful, and is a core component of the Feynman Propagator which is introduced in \autoref{sec:rqft_interactions_feynman-prop}.
\end{document}

