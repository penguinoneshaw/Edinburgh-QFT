% !TeX root = ./notes.tex

\documentclass[notes]{subfiles}
\setcounter{chapter}{3}

\begin{document}

\chapter{Scalar Field Correlators}
\label{cha:scal-field-corr}

\section{Field Correlators}
\label{sec:field-correlators}

As discussed in the previous lecture the field correlators are
obtained from the partition function taking functional derivatives: 
\begin{align}
  \label{eq:CorrDefnPts}
  G^{(n)}\left(x_1, \ldots, x_n\right) 
  &=
    \bra{0} T \phi(x_1) \ldots \phi(x_n) \ket{0}  \\
  \label{eq:CorrDerivGen}
  &= 
    \left(\frac{1}{i} \frac{\delta}{\delta J(x_1)}\right)
    \ldots
    \left(\frac{1}{i} \frac{\delta}{\delta J(x_n)}\right)
    Z[J]\Big|_{J=0} \, .
\end{align}
The perturbative definition of the path integral has led to an
expansion where we can classify terms according to the number of
currents that appear. We denoted this number by $E$ above. Clearly the
only terms that contribute to an $n$-point function are the ones with
$E=n$. For a given values of $E$ there will be contibutions from all
the values of $V$ that satisfy
\[
  E =2P- 3V\, ,
\]
where $P$ in an integer. Therefore the calculation of correlators can
be expressed as a perturbative expansion in powers of the coupling:
\begin{equation}
  \label{eq:CorrPertTh}
  G^{(n)}\left(x_1, \ldots, x_n\right) = 
  \sum_V g^V G^{(n,V)} \left(x_1, \ldots, x_n\right) \, .
\end{equation}
Each functional derivative replaces a factor
of $J$ in the integrand with a Dirac delta. Performing the
corresponding integration leads to replacing the argument at the end
of the propagator that was connected to the current with the argument
of the functional derivative:
\begin{equation}
  \label{eq:DeltaReplace}
  \frac{\delta}{\delta J(x)} \int \ldots d^dy \ldots J(y) \Delta(y,
  \ldots) \ldots =  \int \ldots \Delta(x,\ldots) \ldots\, .
\end{equation}

\subsection{Two-point correlator}
\label{sec:two-point-correlator}

 The two-point function
\begin{align}
  \label{eq:TwoPtOne}
  G^{(2)}(x,y) 
  &= 
    \langle T \phi(x) \phi(y) \rangle \\
  &= 
    \left(\frac{1}{i} \frac{\delta}{\delta J(x)}\right)
    \left(\frac{1}{i} \frac{\delta}{\delta J(y)}\right)
    Z[J]\Big|_{J=0} \, .
\end{align}
The only terms in Eq.~(\ref{eq:DoubleExpExp}) that contribute are the
ones corresponding to $E=2$. 

\paragraph{V=0}

At the lowest order in $g$, \ie for
$V=0$, we have
\begin{equation}
  \label{eq:2ptOrderZero}
    \begin{tikzpicture}[baseline={([yshift=-0.4ex]current bounding box.center)}]
      \begin{feynman}[inline=(a)]
        \vertex (a);
        \vertex (b);
        \diagram {
          a [dot] -- b [dot],
        };
        % \vertex [below=0.2em of a] {\(_{x}\)};  
        % \vertex [below=0.2em of b] {\(_{y}\)};  
      \end{feynman}
    \end{tikzpicture} = \frac{i}{2}
    \int d^Dz_1\, d^Dz_2\, J(z_1) \Delta(z_1-z_2) J(z_2) \, .
\end{equation}
Taking the functional derivatives with respect to $J$ we get a factor
of two, from acting with each derivative on both $J(z_1)$ and
$J(z_2)$:
\begin{equation}
  \label{eq:2ptOrderZeroTwo}
  G^{(2,0)}(x,y) = \frac{1}{i} \Delta(x-y)\, ,
\end{equation}
where the second index in the suffix indicates the order in the
perturbative expansion as discussed above. 

\noindent
It is easy to verify that there are no contributions to $Z[J]$ with
$E=2$ and $V=1$. 

\paragraph{V=2}

The next contributions to the two-point functions
come from terms with $V=2$, and there are two distinct diagram
topologies at this order.
\begin{enumerate}
\item [a.] The first diagram topology that contributes to $G^{(2)}$ is
  \begin{align}
    \begin{tikzpicture}[baseline={([yshift=-0.4ex]current bounding box.center)}]
      \begin{feynman}[inline=(a)]
        \vertex (a);
        \vertex (v1); 
        \vertex (v2); 
        \vertex (b);
        \diagram* {
          a [dot] -- v1 -- [out=90, in=90] v2 -- b [dot],
          v1 -- [out=-90, in=-90]  v2,
        };
      \end{feynman}
    \end{tikzpicture} = &\frac{1}{2^2} \int d^Dz_1\, d^Dz_2\, d^Dw_1\,
      d^Dw_2\, \times \nonumber \\
    &  \times J(z_1) \Delta(z_1-w_1) \Delta(w_1-w_2)^2 \Delta(w_2-z_2) J(z_2)\, .
  \end{align}
  Taking derivatives and inserting the appropriate factor of $i$
  yields a total contribution of
  \begin{align}
    G^{(2,2)}_a(x,y) = - \frac{1}{2} 
    \int d^Dw_1\, d^Dw_2\, \Delta(x-w_1) \Delta(w_1-w_2)^2
    \Delta(w_2-y)\, .
  \end{align}

\item [b.] The other contribution comes from 
  \begin{align}
    \begin{tikzpicture}[baseline={([yshift=-2.95ex]current bounding box.center)}]
      \begin{feynman}[vertical'=i to c]
        \vertex (c);
        \vertex [left=1.7em of c, dot] (a) {}; 
        \vertex [dot, right=1.7em of c](b) {};
        \vertex [above=1.0em of c](i);
%      \vertex (j);
 %     \vertex (k);
      \vertex [above=1.7em of i] (m);
      \diagram* {
%        j -- [draw=none]  i -- [draw=none] k, 
        (i) -- [half left] (m) -- [half left] (i),
%        i -- [out=155, in=25, loop, min distance=3cm] i,
        (a) -- (c) -- (b), 
        (c) -- (i), 
      };
%       \vertex [below=0.2em of a] {\(_{i_1}\)};  
%       \vertex [below=0.2em of b] {\(_{i_2}\)};  
% %      \vertex [below=0.2em of c] {\(c\)};  
%       \vertex [below=0.2em of i] {\(_{i}\)};  
    \end{feynman}
    % \begin{feynman}
    %     \vertex (v1);
    %     \vertex [left=of v1] (a); 
    %     \vertex [above=of v1] (v2); 
    %     \vertex [right=of v1] (b);
    %     \diagram* {
    %       (a) -- (v1)-- (b),
    %       (v1) -- (v2) -- [out=135, in=45, loop, min distance=3cm]  (v2),
    %     };
    %   \end{feynman}
    \end{tikzpicture} = &\frac{1}{2^2} \int d^Dz_1\, d^Dz_2\, d^Dw_1\,
      d^Dw_2\, \times \nonumber \\
    &  \times J(z_1) \Delta(z_1-w_1) \Delta(w_1-z_2) \Delta(w_1-w_2)
      \Delta(0) J(z_2)\, .
  \end{align}
  The net contribution from this diagram is
  \begin{align}
     G^{(2,2)}_b(x,y) = -\frac{1}{2} 
    \int d^Dw_1\, d^Dw_2\, \Delta(x-w_1) \Delta(w_1-y) \Delta(w_1-w_2)
      \Delta(0))\, .
  \end{align}
\end{enumerate}

\subsection{Momentum space}
\label{sec:momentum-space}

It is useful to write this correlators in momentum space. We already
discussed the representation of the free propagator in momentum space: 
\begin{align}
  \Delta(x) = \int \frac{d^Dp}{(2\pi)^D}\, e^{-ip\cdot x}
  \frac{1}{p^2 - m^2 + i\epsilon}\, .
\end{align}
The Fourier transform of the two-point correlator is defined as
\begin{align}
  \tilde{G}^{2}\left(p,p'\right) 
  &= 
    \int d^Dx\, d^Dy\, e^{ip\cdot x} e^{ip'\cdot y} 
    G^{(2)}(x,y) \\
  &= \int d^Dz\, d^Dy\, e^{ip\cdot z} e^{i(p+p')\cdot y} G^{(2)}(z) \\
  &= \left(2\pi\right)^D \delta(p+p')\, \int d^Dz\, e^{ip\cdot z}
    G^{(2)}(z)  \\
  &= \left(2\pi\right)^D \delta(p+p')\, \frac{1}{i} \tilde{\Delta}_F(p)\, .
\end{align}
Note that translation invariance of $G^{(2)}(x,y)$ produces an overall
Dirac delta that implements conservation of the total momentum. Again
we can look at the contributions to $\tilde{G}^{(2)}(p,p')$ order by
order in perturbation theory. 

\paragraph{V=0}

At lowest order in $g$, we simply have the Fourier transform of the
free propagator:
\begin{align}
  \tilde{G}^{(2,0)}(p,p') 
  &= 
    \left(2\pi\right)^D \delta(p+p')\,
    \frac{1}{i} \tilde{\Delta}(p) \\
  &=
    \left(2\pi\right)^D \delta(p+p')\,
    \frac{1}{i} \frac{1}{p^2-m^2+i\epsilon}\, .
\end{align}

\paragraph{V=2}

At order $g^2$, we obtain
\begin{align}
  \tilde{G}^{(2,2)}_a(p,p') 
  =& - \frac12\, \left(2\pi\right)^D \delta(p+p')\,
     \frac{1}{p^2-m^2+i\epsilon} \times \nonumber \\
   & \times \left\{
     \int \frac{d^D\ell}{(2\pi)^D}\, 
     \frac{1}{\ell^2-m^2+i\epsilon}
     \frac{1}{(\ell-p)^2-m^2+i\epsilon}
     \right\}
     \frac{1}{p^2-m^2+i\epsilon} \, .
\end{align}
The above expression can be represented by a Feynman diagram in
momentum space:
\begin{equation}
  \begin{tikzpicture}%[baseline=\plusheight]
    \begin{feynman}
      \vertex (a);
      \vertex [left=0.7cm of a] (i1);
      \vertex [right= of a] (b);
      \vertex [right=0.7cm of b] (f1);
      \diagram [horizontal=a to b, layered layout] {
        (i1) -- [momentum={\(p\)}] (a)
        -- [half left, momentum={[arrow shorten=0.4]\(\ell\)}] (b) 
        -- [half left, momentum={[arrow shorten=0.35]\(\ell-p\)}] (a) ,
        (f1) -- [momentum'=\(p'\)] (b)
      };
    \end{feynman}
  \end{tikzpicture}
\end{equation}
The Feynman rules in momentum space can be summarised as follows
(adapted from Srednicki's book). 
\begin{enumerate}
\item Draw $n$ lines for a $n$-point correlator. 
\item Leave one end of each external line free, and attach the other
  to the lines coming out of a vertex. 
\item The $i$-th external line carries momentum $p_i$, which we assume
  to be incoming momentum, and represent with a line pointing towards
  the vertex.
\item Four-momenta flow along the arrows, and the total momentum is
  conserved at each vertex. For a diagram without loops, this fixes
  the momentum of \emph{all} internal lines. 
\item The value of the diagram is given by the product of a factor of
  $i/(p^2-m^2+i\epsilon)$ for each line with momentum $p$, a factor of
  $1/i$ for the external end of a line, and a factor $ig (1/i)^\#$ for
  each vertex, where $\#$ is the number of legs connecting at each
  vertex. 
\item A diagram with $L$ loops will have $L$ internal momenta that are
  not fixed by momentum conservation. We integrate over those momenta,
  with measure $d^p/(2\pi)^D$. 
\item Determine the symmetry factor associated to permutations of
  \emph{internal} propagators and vertices. 
\end{enumerate}

\begin{Ex}
  Use the Feynman rules in momentum space to compute
  $G^{(2,2)}_b$. Check that you get the same result by performing a
  Fourier transform of the result in position space.
\end{Ex}

\section{Physical states}
\label{sec:physical-states}

The eigenstates of the Hamiltonian form a complete set of states. They
can be classified in three categories.
\begin{enumerate}
\item The vacuum state $\ket{0}$ is the lowest energy state, and
  corresponds to a state with no particles.
\item The one-particle states $|\mathbf{p},\sigma\rangle$ are
  classified by their spatial momentum. Their energy is given by the
  relativistic dispersion relation 
  \begin{equation}
    \label{eq:RelDispRel}
    E_p = \sqrt{\mathbf{p}^2 + \mphys^2}\, ,
  \end{equation}
  where $\mphys$ denotes the physical mass of the state. Note that the
  physical mass \emph{does not} need to coincide with the mass that
  appears in the Lagrangian. We will discuss this point in detail
  later. Any other quantum number necessary to identify the particle
  is denoted here by $\sigma$.  The states are normalised by imposing
\begin{equation}
  \label{eq:OnePartNorm}
  \langle \mathbf{p}, \sigma | \mathbf{p}', \sigma' \rangle =
  \delta_{\sigma\sigma'}\, 
  2 E_p\, (2\pi)^{D-1} \delta(\mathbf{p}-\mathbf{p}')\, .
\end{equation}
\item The multiparticle states $|\mathbf{P};n\rangle$ are classified
  by their total spatial momentum $\mathbf{P}$, plus other parameters
  such as the relative momenta between the particles, which we denote
  collectively by $n$. The energy of the mutiparticle states is
  $\sqrt{\mathbf{P}^2+M^2}$, where $M^2$ is one of the parameters
  included in $n$. The threshold for producing multiparticle states is
  given by the energy of two particles at rest $M=2\mphys$.
\end{enumerate}
The completeness relation can be written as
\begin{align}
  \ketbra{0}{0}
  &+ \sum_\sigma \int d\Omega_p 
    |\mathbf{p},\sigma\rangle \langle \mathbf{p},\sigma| + \nonumber
  \\
  &+ \sum_n \int d\Omega_P 
    |\mathbf{P},n\rangle \langle \mathbf{P},n|\, = 1\, ,
\end{align}
and we have introduced the Lorentz-invariant integration measure
\begin{align}
  d\Omega_p = \frac{d^{D-1}p}{(2\pi)^{D-1}}
    \frac{1}{2E_p} \, .
\end{align}
Note that the 'sum' over $n$ is a short-hand notation, which may
involve integrations over continuum variables such as the relative
momentum, or the invariant mass of the state. 

\section{Polology}
\label{sec:polology}

In this section, we shall learn some features about the analytic
structure of field correlators, and discuss their relevance in order
to extract physical information from the correlators. Starting from an
$n$-point correlator in momentum space,
\begin{align}
  \tilde{G}^{(n)}\left(p_1, \ldots, p_n\right) 
  = \int d^Dx_1\, \ldots d^Dx_n\, 
  e^{-i p_1\cdot x_1} \ldots e^{-i p_n\cdot x_n}\, 
  \langle T\phi(x_1) \ldots \phi(x_n)\rangle\, ,
\end{align}
we want to focus on the contribution coming from the sector where the
values $x_1^0, \ldots, x_r^0$ are all larger than the values of
$x_{r+1}^0, \ldots, x_n^0$, for some value of $r$ between 1 and
$n-1$. This contribution can be written as 
\begin{align}
  \tilde{G}^{(n)}\left(p_1, \ldots, p_n\right) 
  =& \int d^Dx_1\, \ldots d^Dx_n\, 
  e^{-i p_1\cdot x_1} \ldots e^{-i p_n\cdot x_n}\, \times \nonumber \\
   &\times \theta\left(
     \min\{x_1^0 \ldots x_r^0\} - \max\{x_{r+1}^0 \ldots x_n^0\}
     \right) \nonumber \\
   &\times \langle T\Big[\phi(x_1) \ldots \phi(x_r)\Big]
     T\Big[\phi(x_{r+1}) \ldots \phi(x_n)\Big]
     \rangle\, .
\end{align}
We can introduce a complete set of states in-between the two
$T$-ordered products, and look at the result coming from the
one-particle states. Defining new integration variables
\begin{align}
  x_i = x_1 + y_i\, , \quad \mathrm{for}\ i=2, \ldots, r\, ,
\end{align}
we can write
\begin{align}
  \phi(x_1) \ldots \phi(x_r) 
  &= \phi(x_1) \ldots \phi(x_1+y_i) \ldots \phi(x_1+y_r) \nonumber \\
  &= e^{i P\cdot x_1} \phi(0) e^{-i P\cdot x_1} \ldots e^{i P\cdot
    x_1} \phi(y_i) e^{-i P\cdot x_1} \ldots e^{i P\cdot x_1} \phi(y_r)
    e^{-i P\cdot x_1}\, ,
\end{align}
and hence
\begin{align}
  \bra{0} T \phi(x_1) \ldots \phi(x_r) | p,\sigma\rangle = 
  e^{-i p \cdot x_1}  \bra{0} T \phi(0) \ldots \phi(y_r) |
  p,\sigma\rangle\, .
\end{align}
A similar shift of the integration variables can be done using
\begin{align}
  x_i = x_{r+1} + y_i\, , \quad \mathrm{for}\ i=r+2, \ldots, n\, ,
\end{align}
The argument of the theta function can be rewritten as
\begin{align}
  \min\{x_1^0 \ldots x_r^0\} - \max\{x_{r+1}^0 \ldots x_n^0\} = 
  x_1^0 -x_{r+1}^0 + \min\{0 \ldots y_r^0\} - \max\{0 \ldots y_n^0\}\, .
\end{align}
Using the integral representation of the theta,
\begin{align}
  \theta(\tau) = -\frac{1}{2\pi i} \int_{-\infty}^{+\infty} d\omega\,
  \frac{e^{-i\omega \tau}}{\omega + i\epsilon}\, ,
\end{align}
and performing the integrals over $x_1$ and $x_{r+1}$, yields
\begin{align}
  \tilde{G}^{(n)}(p_1, \ldots, p_n) 
  &= \int d^Dy_2 \ldots d^Dy_r\, d^Dy_{r+2} \ldots d^Dy_n \, \nonumber
  \\
  & \times e^{-ip_2\cdot y_2} \ldots e^{-ip_{r}\cdot y_r} 
    \, e^{-ip_{r+2}\cdot y_{r+2}} \ldots e^{-ip_{n}\cdot y_n}
    \nonumber \\
  & \times \frac{-1}{2\pi i} \int \frac{d\omega}{\omega + i\epsilon}\, 
    \exp\left\{
    -i \omega \left[ 
    \min\{0 \ldots y_r^0\} - \max\{0 \ldots y_n^0\}
    \right]
    \right\} \nonumber \\
  & \times \sum_\sigma \int d\Omega_p
    \bra{0}  T \phi(0) \ldots \phi(y_r) |
    p,\sigma\rangle \langle p, \sigma | T \phi(0) \ldots \phi(y_n)
    \ket{0} \nonumber \\
  & \times (2\pi)^D \delta(\mathbf{p} - \mathbf{p}_1 - \ldots -
    \mathbf{p}_r) \, 
    \delta(E_p + \omega - p_1^0 - \ldots - p_{r}^0) \nonumber \\
  & \times (2\pi)^D \delta(\mathbf{p} + \mathbf{p}_{r+1} + \ldots +
    \mathbf{p}_n) \, 
    \delta(E_p + \omega + p_{r+1}^0 + \ldots + p_{n}^0) \, .
\end{align}
Performing the integrals over the spatial components of $p$ and
$\omega$ yields
\begin{align}
  &\delta(\mathbf{p}_1+ \ldots +\mathbf{p}_n)\, \mathrm{and}\, , \nonumber \\
  & \delta(p_1^0 + \ldots + p_n^0) \frac{1}{q^0-E_p +i\epsilon}\, ,
\end{align}
respectively, where the four-momentum $q$ is defined as
\begin{align}
  q=p_1 + \ldots + p_r = -p_{r+1} - \ldots - p_n\, .
\end{align}
Finally, if we are interested in the residue at the pole, we can rewrite
\begin{align}
  \frac{1}{q^0-E_p + i\epsilon} \longrightarrow \frac{2 E_p}{q^2-\mphys^2 + i\epsilon}\, 
\end{align}
Collecting all the terms, and ignoring a phase factor that reduces to
one at the pole, we find that the one-particle state contribution to the
integration over the specific sector that we considered above yields
\begin{align}
  \tilde{G}^{(n)}(p_1,\ldots,p_n) \longrightarrow
  \delta(p_1+\ldots+p_n)\, \frac{1}{q^2-\mphys^2+i\epsilon}\, 
  \sum_\sigma M_{0|q\sigma}(p_2,\ldots,p_r) M_{q\sigma|0}(p_{r+2},
  \ldots, p_n)\, .
\end{align}
In the equation above we have defined
\begin{align}
  M_{0|q\sigma}(p_2,\ldots,p_r) 
  &= \int d^Dy_2 \ldots d^Dy_r\, 
    e^{-ip_2\cdot y_2} \ldots e^{-ip_r\cdot y_r}
    \mel{0}{T \phi(0) \ldots \phi(y_r)}{q,\sigma}, \\
 M_{q\sigma|0}(p_{r+2},\ldots, p_n) 
  &= \int d^Dy_{r+2} \ldots d^Dy_n\, 
    e^{-ip_{r+2}\cdot y_{r+2}} \ldots e^{-ip_n\cdot y_n}
    \mel{q,\sigma}{  T \phi(0) \ldots \phi(y_n)}{0},
\end{align}
and we notice that this contribution appears multiplied by a Dirac
delta that enforces the conservation of total momentum. The important
result here is that the correlators in momentum space have a pole
singularity whenever $q=p_1+\ldots+p_r$ \emph{goes on-shell}, \ie
$q^2=\mphys^2$.

Note that these results are completely general, and in particular do
not rely on the perturbative definition of the correlators. 

\section{Källén-Lehmann representation}
\label{sec:kall-lehm-repr}

Let us now come back to the 2-point function, and find a representation
that allows us to extract some physical information about the scalar
field. We define the full propagator as
\begin{equation}
  \label{eq:FullProp}
  \Delta_F(x-y) = i \bra{0} T \phi(x) \phi(y) \ket{0} \, ,
\end{equation}
and define the field so that
\begin{equation}
  \label{eq:FieldNorm}
  \bra{0} \phi(x) \ket{0} = 0\, ,\ \mathrm{and}\
  \langle \mathbf{p}| \phi(0) \ket{0} = 1\, ,
\end{equation}
where $\ket{\mathbf{p}}$ represents the physical one-particle state,
and we have dropped the dependence on $\sigma$. As usual the full
propagator in momentum space is defined by taking the Fourier
transform:
\begin{equation}
  \label{eq:FullPropMom}
  \tilde{\Delta}_F(p) = 
  \int d^Dx\,  e^{ip\cdot(x-y)} \Delta(x-y)\, .
\end{equation}

\paragraph{Free theory}

With these conventions, the free theory result for the propagator is
\begin{equation}
  \label{eq:FreePropAgain}
  \tilde\Delta(p) = \frac{1}{p^2-m^2+i\epsilon}\, .
\end{equation}
Eq.~(\ref{eq:FreePropAgain}) shows that $\tilde{\Delta}$ has a pole at
$p^2=m^2$. For the free particle we find a pole in the propagator,
at a value which coincides with the parameter in the Lagrangian.

\paragraph{Interacting theory}

For the interacting theory, we can derive a general expression, which
again does not rely on the perturbative definition of the two-point
function. Let us first consider the case where $x^0>y^0$:
\begin{align}
  \bra{0} T \phi(x) \phi(y) \ket{0}
  =& \bra{0} \phi(x) \phi(y) \ket{0} \nonumber \\
  =& \bra{0} \phi(x) \ket{0} \bra{0} \phi(y) \ket{0} +
     \nonumber \\
  &+ \int d\Omega_p\, \bra{0} \phi(x)| \mathbf{p}\rangle
    \langle \mathbf{p} | \phi(y) \ket{0} + \nonumber \\
  &+ \sum_n \int d\Omega_P\, \bra{0} \phi(x) |\mathbf{P}, n\rangle
    \langle \mathbf{P}, n |\phi(y) \ket{0}\, .
\end{align}
The conventions in Eq.~(\ref{eq:FieldNorm}) allow us to simplify the
expression above.
\begin{align}
  \bra{0} T \phi(x) \phi(y) \ket{0}
  =& \int d\Omega_p\, e^{-ip\cdot (x-y)} + \nonumber \\
   &+ \sum_n \int d\Omega_p\, e^{-ip\cdot (x-y)}\, \left|
    \langle \mathbf{p}, n |\phi(0) \ket{0} \right|^2 \, .
\end{align}
Because we are working with a scalar field, the matrix element
$\langle \mathbf{p}, n |\phi(0) \ket{0}$ is invariant under Lorentz
transformations, and therefore can only depend on $p$ via the
invariant mass $M^2$. We can therefore introduce the \emph{spectral
  density}
\begin{align}
  \label{eq:SpecDen}
  \rho(s) = \sum_n \left|
    \langle \mathbf{p}, n |\phi(0) \ket{0} \right|^2 \, 
  \delta(s-M^2)\, ,
\end{align}
and write the two-point correlator as
\begin{align}
  \bra{0} T \phi(x) \phi(y) \ket{0}
  =& \int d\Omega_p\, e^{-ip\cdot (x-y)} + \nonumber \\
   &+ \int_{4m^2}^\infty ds\, \rho(s) \int d\Omega_p\,  e^{-ip\cdot
     (x-y)}\, .
\end{align}
Similar manipulations for the case $y^0>x^0$ yield
\begin{align}
  \bra{0} T \phi(x) \phi(y) \ket{0}
  =& \int d\Omega_p\, e^{ip\cdot (x-y)} + \nonumber \\
   &+ \int_{4m^2}^\infty ds\, \rho(s) \int d\Omega_p\,  e^{ip\cdot
     (x-y)}\, .
\end{align}
Collecting both contributions to the $T$-ordered product 
\begin{align}
  \bra{0} T \phi(x) \phi(y) \ket{0}
  = \theta(x^0-y^0)  \bra{0} \phi(x) \phi(y) \ket{0}
  + \theta(y^0-x^0)  \bra{0} \phi(y) \phi(x) \ket{0}\, ,
\end{align}
and using 
\begin{align}
  \frac{1}{i} \int \frac{d^Dp}{(2\pi)^D}\,
  \frac{e^{-ip\cdot(x-y)}}{p^2-\mphys^2+i\epsilon} = 
  \theta(x^0-y^0) \int d\Omega_p\, e^{-ip\cdot(x-y)} +
  \theta(y^0-x^0) \int d\Omega_p\, e^{ip\cdot(x-y)}\, , 
\end{align}
we finally obtain
\begin{align}
  \bra{0} T \phi(x) \phi(y) \ket{0}
  =& \int \frac{d^Dp}{(2\pi)^D}\,
     e^{-ip\cdot(x-y)} \Big[
     \frac{1}{p^2-\mphys^2+i\epsilon} + \nonumber \\
  \label{eq:KLFourier}
  & + \int_{4\mphys^2}^\infty ds\, \rho(s) \frac{1}{p^2-s+i\epsilon} 
     \Big]\, .
\end{align}
Eq.~(\ref{eq:KLFourier}) allows us to read the expression for the full
propagator in momentum space:
\begin{align}
  \label{eq:KLusual}
  \tilde{\Delta}_F(p) = \frac{1}{p^2-\mphys^2+i\epsilon} 
  + \int_{4\mphys^2}^\infty ds\, \rho(s) \frac{1}{p^2-s+i\epsilon}\, . 
\end{align}
We see that the two-point correlator of a field $\phi$ that satisfies
the conditions in Eq.~(\ref{eq:FieldNorm}) has a pole for
$p^2=\mphys^2$, with residue exactly equal to one. Note that the field
$\phi$ does not need to be the field that appears in the
Lagrangian. Knowledge about the multiparticle states of the theory is
encoded in the two-point function via the integral on the RHS side of
Eq.~(\ref{eq:KLusual}). 

\section{S Matrix}
\label{sec:s-matrix}

In order to compute the quantum amplitude for a physical process
involving arbitrary numbers of particles in the initial and final
state, we need to compute the overlap of a state prepared in the
distant past (the so-called \emph{in} state), with the resulting final
state in the distant future (the so-called \emph{out} state). If
we want to describe a $2\to n$ process -- like a $pp$ collision at
the LHC -- we need to compute
\begin{equation}
  \label{eq:ScattAmpl}
  \langle \mathbf{p}_1, \ldots, \mathbf{p}_n; \mathrm{out} |
  \mathbf{k}_1, \mathbf{k}_2; \mathrm{in}\rangle\, .
\end{equation}
The $S$-matrix allows us to express this scalar product between in-
and out-states in terms of states defined at any common reference
time: 
\begin{equation}
  \label{eq:SMatDef}
   \langle \mathbf{p}_1, \ldots, \mathbf{p}_n; \mathrm{out} |
  \mathbf{k}_1, \mathbf{k}_2; \mathrm{in}\rangle = 
   \langle \mathbf{p}_1, \ldots, \mathbf{p}_n | S |
  \mathbf{k}_1, \mathbf{k}_2\rangle\, .
\end{equation}
It is usual to separate the $S$ matrix into the identity operator,
corresponding to particles not interacting, plus a non-trivial part
which is usually denoted $T$:
\begin{equation}
  \label{eq:TMatDef}
  S = 1 + i T\, .
\end{equation}

\section{LSZ reduction}
\label{sec:lsz-reduction}

An important corollary of the result shown in
section~\ref{sec:polology} is obtained by setting $r=1$. In this case,
the previous discussion allows us to conclude that the correlators in
momentum space have a pole whenever the momentum of one of the fields
is on-shell. Therefore an $n$-point correlation function has (at
least) $n$ poles, each corresponding to one of the momenta
$p_i\to \mphys^2$. The residue at this multiple pole yields the $S$-matrix
for a scattering process involving $n$-particles:
\begin{align}
   \braket{ p_{1}' \ldots p_{m'}'; \mathrm{out} }{ p_{1} \ldots p_{m};
  \mathrm{in}}
  =& 
  \mel{p_{1}' \ldots p_{m'}'}{S}{p_{1} \ldots p_{m}} \\
  =& 
  \lim_{p_j^2,p_k'^2 \to \mphys^2} \prod_{k=1}^{m'}
  (p_k'^2-\mphys^2+i\epsilon)
  \prod_{j=1}^{m}
  (p_j^2-\mphys^2+i\epsilon) \, \nonumber \\
  & \times 
    \tilde{G}^{(m+m')}(p_1, \ldots, p_{m}, -p_1', \ldots, -p_{m'}')\, ,
\end{align}
where $n=m+m'$, and the fields are normalised so that
\begin{align}
  \langle \mathbf{p} | \phi(0) \ket{0} = 1\,.
\end{align}
We will return to the question of the normalisation of the field
later. The LSZ reduction formula provides an elegant way to represent
quantum amplitudes using Feynman diagrams in momentum space. We adopt
the same rules discussed above, with the following modifications.
\begin{enumerate}
\item We associate an outgoing momentum to the external lines that
  correspond to particles in the final state. 
\item We multiply each external line by a factor of
  $-i(p^2-\mphys^2+i\epsilon)$ -- the correlators multiplied by these
  factors are called \emph{truncated} (or \emph{amputated}) correlators.
\end{enumerate}

A heuristic derivation of the LSZ reduction formula is discussed in
Problem Sheet 4.

\begin{Ex}
  Compute the amplitude for the scattering process
  \[
    p_1 p_2 \longrightarrow p_1' p_2'
  \]
  at order $g^2$ in the $\phi^3$ scalar theory. You can assume that
  $\mphys=m$ in this calculation.
\end{Ex}

\section{Optical Theorem}
\label{sec:optical-theorem}

Physical constraints translate into relations between correlators. It
is important to be able to derive these relations, and to understand
their physical content. One example is provided by the unitarity of
the $S$-matrix, \ie by the conservation of probability in quantum
mechanics. Unitarity is written as
\begin{equation}
  \label{eq:SmatUnit}
  S^\dagger S = 1\, .
\end{equation}
Inserting the representation of $S$ in terms of the transition matrix
yields
\begin{equation}
  \label{eq:TmaUnit}
  -i \left(T - T^\dagger\right) = T^\dagger T\, .
\end{equation}
Let us consider the matrix element of Eq.~(\ref{eq:TmaUnit}) between an
initial state $a$ and a final state $b$, and let us factor out a Dirac
delta that corresponds to total momentum conservation, 
\begin{equation}
  \label{eq:MAmplitude}
  \bra{b} T \ket{a} = (2\pi)^D \delta(P_a - P_b)\, \mathcal{M}(a
  \to b)\, .
\end{equation}
Some simple algebra yields on the LHS
\begin{equation}
  \label{eq:MLHS}
  -i (2\pi)^D \delta\left(P_a-P_b\right) \left[
    \mathcal{M}(a\to b) - \mathcal{M}(b \to a)^*
    \right]\, . 
\end{equation}
On the RHS we can insert a complete set of states, and rewrite it as
\begin{equation}
  \label{eq:MRHS}
   (2\pi)^D \delta\left(P_a-P_b\right) \sum_f \int d\Omega_f\,
   (2\pi)^D \delta\left(P_a-P_f\right) \mathcal{M}(b\to f)^*
   \mathcal{M}(a\to f)\, .
 \end{equation}
The unitarity condition simplifies for $a=b$, 
\begin{equation}
  \label{eq:OptThm}
  2 \mathrm{Im}\ \mathcal{M}(a\to a) = 
  \sum_f \int d\Omega_f  (2\pi)^D \delta\left(P_a-P_f\right) 
  \left|
    \mathcal{M}(a\to f)
  \right|^2\, .
\end{equation}
This is the so-called \emph{optical theorem}, which relates the
imaginary part of the forward $a\to a$ amplitude (LHS) to the total cross
section $a\to f$ (RHS), summed over \emph{all} final states $f$.

\section{Ward identities}
\label{sec:ward-identities}

The final example of relations between correlators that we are going
to discuss are the so-called \emph{Ward identities}. Ward identities
are equalities between field correlators that are obtained as a
consequence of symmetries of the system. In classical mechanics,
symmetries of the action translate into conserved currents according
to Noether's theorem. As we will show in this section, the analogue of
current conservation in quantum field theory is precisely the Ward
identity. 

In order to derive the identities, let us start by considering a
symmetry transformation of the field, \ie a transformation
\begin{equation}
  \label{eq:FieldTrans}
  \phi(x) \mapsto \phi'(x)=\phi(x) + \epsilon \delta\phi(x) \, ,
\end{equation}
such that for constant $\epsilon$ the action is unchanged. If we introduce
a dependence on the space-time coordinate, $\epsilon(x)$, then the
variation of the action can be written
\begin{align}
  \label{eq:SVar}
  \delta S &= \int d^Dx\, \frac{\delta S}{\delta\phi(x)} \epsilon(x)
             \delta\phi(x) \\
             &= -\int d^Dx\, \epsilon(x) \partial_\mu j^\mu(x)\, , 
\end{align}
where $j^\mu(x)$ is precisely the Noether current that is conserved in
the classical theory. 

In order to derive the Ward identities, we use
Eq.~(\ref{eq:FieldTrans}) to perform a change of integration variables
in the functional integral
\begin{equation}
  \label{eq:ChangeOfVars}
  \int \mathcal{D}\phi\, e^{iS[\phi]} O(\phi) = 
  \int \mathcal{D}\phi'\, e^{iS[\phi']} O(\phi') \, ,
\end{equation}
and then expand the RHS to first order in $\epsilon$:
\begin{align}
  \int \mathcal{D}\phi\, e^{iS[\phi]} O(\phi) 
  =& \int \mathcal{D}\phi\, e^{iS[\phi]}\, 
     \left[
     1 + i \delta S[\phi]
     \right]\, 
     \left[
     O(\phi) + \delta O
     \right]
     \, ,
\end{align}
where $O$ is a generic function of the field $\phi$. We can now
substitute the expressions for $\delta S$ and $\delta O$:
\begin{align}
  \int \mathcal{D}\phi\, e^{iS[\phi]} \left\{
  -i \int d^Dx\, \epsilon(x) \partial_\mu j^\mu(x)
  O(\phi) + \int d^Dx\, \frac{\delta O(\phi)}{\delta\phi(x)} \epsilon(x)
  \delta \phi(x)
  \right\} = 0\, .
\end{align}
Rearranging the terms above allows us to write the identity in a way
that makes its physical content more obvious: 
\begin{align}
  \label{eq:IntWardId}
  \int d^Dx\, \epsilon(x) \left\{
  -i \langle \partial_\mu j^\mu(x)
  O(\phi) \rangle +  
  \langle \frac{\delta O(\phi)}{\delta\phi(x)}
  \delta \phi(x) \rangle \right\} = 0\, .
\end{align}
Eq.~(\ref{eq:IntWardId}) is sometimes referred to as an {\em
  integrated Ward identity}. Since it has to be satisfied for every
function $\epsilon(x)$, we can derive the \emph{Ward identity}:
\begin{align}
  \label{eq:WardId}
   -i \langle \partial_\mu j^\mu(x)
  O(\phi) \rangle +  
  \langle \frac{\delta O(\phi)}{\delta\phi(x)}
  \delta \phi(x) \rangle = 0\, .
\end{align}
There are two important physical results encoded in Eq.~(\ref{eq:WardId}).
\begin{enumerate}
\item Symmetry in QFT translates into a relation between
  correlators. This is true beyond perturbation theory and is used in
  defining the renormalization conditions in QFT.
\item Current conservation in QFT is realised at the level of the
  insertion of $\partial_\mu j^\mu(x)$ in field correlators, up to
  the terms that come from the variation of $O$. If $O$ is a product
  of local fields, this variations is localised in space-time, \ie the
  contributions are all proportional to Dirac deltas. These terms are
  called \emph{contact terms}.  
\end{enumerate}

Note that in deriving the Ward identity above we have assumed that the
integration measure $\mathcal{D}\phi$ is invariant, \ie
$\mathcal{D}\phi=\mathcal{D}\phi'$. There are examples where the
measure is \emph{not} invariant, which lead to extra terms in the Ward
identities. In these cases the Ward identities are called {\em
  anomalous}. 
\end{document}