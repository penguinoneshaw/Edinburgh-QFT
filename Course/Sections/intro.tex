%!TeX root = ../notes.tex
\documentclass[notes]{subfiles}
  
\begin{document}
\chapter*{Introductory Thoughts}
\label{chap:intro}
This is introductory material. It is split into the `relativistic' (or canonical) and `modern' (path integral based) halves of the course, with the first part adapted from the notes as lectured by Richard Ball, and the second written by Luigi Del Debbio.
As such, there's somewhat a stylistic shift between the two sections.

\section*{Convention}
By the typical convention for quantum field theory, we work in natural units in which $c = \hslash = 1$, and use the mostly-negative convention for the Minkowski metric, with 
\begin{equation*}
    \eta_{\mu\nu} = \begin{pmatrix}
        \dmat[0]{1, -1, -1, -1}
    \end{pmatrix}.
\end{equation*}
\end{document}