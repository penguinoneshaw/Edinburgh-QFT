\renewcommand{\tphi}{\tilde{\phi}}
\renewcommand{\tj}{\tilde{J}}
\renewcommand{\tchi}{\tilde{\chi}}
\renewcommand{\psibar}{\bar{\psi}}
\renewcommand{\etabar}{\bar{\eta}}
\renewcommand{\munu}{{\mu\nu}}



\section{A first look at divergences}
\label{sec:a-first-look}

In this lecture we will aim to develop a self-consistent treatment of
divergences in QFT. This is a vast topic, which can hardly be
addressed exhaustively in the time that we have. Therefore we will
follow the following steps.
\begin{enumerate}
\item Compute the scalar two-point function beyond the first order in
  perturbation theory. As we try to perform this calculation we will
  encounter our first divergent integral.
\item Discuss the regularization of divergences; \ie a procedure that
  allows us to manipulate well-defined mathematical expressions, and
  to identify the structure of the divergences.
\item Discuss the renormalization of divergences; \ie the conditions
  that are necessary for a quantum field theory to produce finite,
  unambiguous predictions. 
\end{enumerate}

\subsection{Two-point function in perturbation theory}
\label{sec:scalar-two-point}

Working in perturbation theory, we compute the two-point function
\begin{equation}
  \label{eq:TwoPtMom}
  \tilde{G}^{(2)}\left(p,p'\right) =
  (2\pi)^D \delta\left(p+p'\right) \frac{1}{i} \tilde{\Delta}_F(p)\, ,
\end{equation}
as a Taylor expansion in powers of the coupling constant
\begin{equation}
  \label{eq:TwoPtMomPert}
  \tilde{G}^{(2)}\left(p,p'\right) = \sum_k g^k
  \tilde{G}^{(2,k)}\left(p,p'\right)\, .
\end{equation}
As discussed before, the delta function in Eq.~\ref{eq:TwoPtMom}
ensures momentum conservation. For all practical purposes, we should
remember that it is there, and work on the perturbative expansion of
the full propagator 
\begin{equation}
  \label{eq:PropMomPert}
  \tilde{\Delta}_F(p) = \sum_k g^k \tilde{\Delta}^{(k)}_F(p)\, .
\end{equation}
From our previous computations
\begin{align}
   \frac{1}{i} \tDelta_F^{(2)}(p)
  &= -\frac{1}{2} \SProp{p} \left(\int \frac{d^D\ell}{(2\pi)^D}\,
    \SProp{\ell} \SProp{(\ell-p)}\right) \times \nonumber \\
  &\quad \times \SProp{p}\, .,
\end{align}
and therefore the $O(g^2)$ contribution to the correlator can be
written as
\begin{equation}
  \label{eq:OneLoopPi}
  \frac{1}{i} \tDelta(p) \left(i\Pi(p^2)\right) \frac{1}{i}
  \tDelta(p)\, ,
\end{equation}
where
\begin{equation}
  \label{eq:PiFun}
  i\Pi(p^2) = \frac{g^2}{2} \int  \frac{d^D\ell}{(2\pi)^D}\,
  \SProp{\ell} \SProp{(\ell-p)}\, .
\end{equation}

\subsection{Evaluation of $\Pi(p^2)$}
\label{sec:evaluation-pi}

\paragraph{Feynman parameters}

The product of propagators in Eq.~(\ref{eq:PiFun}) can be rewritten
using Feynman parameters. The general formula
\begin{align}
  \frac{1}{A_1^{\alpha_1} \ldots A_n^{\alpha_n}}
  = & \frac{\Gamma(\alpha_1+\ldots +\alpha_n)}{\Gamma(\alpha_1)\ldots
      \Gamma(\alpha_n)}\, \int_0^1 dx_1\, x_1^{\alpha_1 -1} \ldots
      \int_0^1 dx_n\, x_n^{\alpha_n -1} \nonumber \\
    & \quad \times \delta\left(1-x_1-\ldots -x_n\right)\,
      \frac{1}{\left(x_1 A_1 + \ldots + x_n
      A_n\right)^{\alpha_1 + \ldots + \alpha_n}}\, , 
\end{align}
can be applied to the integrand above, yielding
\begin{align}
  \SProp{\ell} \SProp{(\ell-p)} = \int_0^1dx\,
  \frac{1}{\left(q^2-M^2 +i\epsilon\right)}\, ,
\end{align}
where $q=\ell-xp$, and $M^2(x,p)=m^2-x(1-x)p^2$. Hence, we have
\begin{equation}
  \label{eq:PiAfterFeynPar}
  i\Pi(p) = \frac{g^2}{2} \int_0^1dx\, \int \frac{d^Dq}{(2\pi)^D}\, 
  \frac{1}{\left(q^2-M^2 +i\epsilon\right)^2}\, .
\end{equation}

\paragraph{Wick rotation}

It is useful to introduce Euclidean momenta in order to perform the
integration. Because of the location of the poles, we can rotate the
integration contour clockwise by $\pi/2$ to run along the purely
imaginary axis. Introducing
\begin{equation}
  \label{eq:EuclideanMom}
  q^0 = i q^0_E\, , \quad \mathbf{q} = \mathbf{q}_E\, ,
\end{equation}
we can rewrite
\begin{equation}
  \label{eq:PiEuclMom}
  \Pi(p^2) = \frac{g^2}{2} \int_0^1dx\, \int \frac{d^Dq_E}{(2\pi)^D}\, 
  \ESProp{q_E}{2}\, .
\end{equation}

\paragraph{A comment on divergences}

The integral in Eq.~(\ref{eq:PiEuclMom}) is clearly divergent for
$D\geq 4$. In particular, it is quadratically divergent in the UV for
the case $D=6$, which is the one we will be interested in. Before
developing more sophisticated tool, we can make a simple, but rather
deep, observation. If we take the derivative of $\Pi(p^2)$ with
respect to $p^2$:
\begin{equation}
  \label{eq:PiPrimeInt}
  \Pi'(p^2) = -g^2 \int_0^1dx\, x(x-1) \int \frac{d^Dq_E}{(2\pi)^D}\,
  \ESProp{q_E}{3}\, ,
\end{equation}
which is still divergent, but only for $D\geq 6$. Similarly
\begin{equation}
  \label{eq:PiDoublePrimeInt}
  \Pi''(p^2) = 3g^2 \int_0^1dx\, x^2(x-1)^2 \int \frac{d^Dq_E}{(2\pi)^D}\,
  \ESProp{q_E}{4}
\end{equation}
is divergent only for $D\geq 8$, and in particular is finite for
$D=6$. Therefore the function $\Pi(p^2)$ can be reconstructed by
integrating its second derivative twice,
\begin{equation}
  \label{eq:IntTwicePi}
  \Pi(p^2) = \Pi(\mu_1^2) + \Pi'(\mu_2^2) \left(p^2-\mu_1^2\right) +
  \int_{\mu_1^2}^{p^2} ds'\, \int_{\mu_2^2}^{s'} ds\, \Pi''(s)\, .  
\end{equation}
Eq.~(\ref{eq:IntTwicePi}) shows clearly that $\Pi(p^2)$ in $D=6$ is well
defined for all values of $p^2$ provided we fix the values of
$\Pi(\mu_1^2)$, and $\Pi'(\mu_2^2)$, \ie the values of the function
and its derivative at two {\em arbitrary} values of the scale.

\section{Regularization}
\label{sec:regularization}

In order to make progress in our understanding of these divergences,
we need to first regulate the theory, \ie we need to choose a
prescription that makes the loop integrals mathematically well
defined. This is clearly a necessary condition in order to be able to
manipulate these expressions, and eventually define a predictive
theory that yields finite results for physical quantities.

There are several ways of regularizing the theory, here we list some
of the most common procedures, some of which we will explore in
tutorials.
\begin{enumerate}
\item Sharp cutoff in Euclidean momenta, $q_E^2\leq \Lambda^2$.
\item Pauli-Villars regulator. The propagators are modifed in order to
  have a less divergent behaviour at large values of the momenta:
  \begin{equation}
    \label{eq:PVReg}
    \SProp{p} \mapsto \SProp{p} - \frac{1}{p^2-M^2+i\epsilon}\, ,
  \end{equation}
  where $M$ is a mass scale that plays the role of the UV cutoff.
\item Schwinger-time regularization -- see PS7.
\item Work in generic dimension $D$, and define the divergent
  integrals by analytical continuation. 
\end{enumerate}

\subsection{Dimensional Regularization}
\label{sec:dimens-regul}

In dimensional regularization (DimReg), loop integrals are computed in
a generic dimension $D$, where the integration is actually
convergent, and then defined for arbitrary values of $D$ by analytical
continuation. Let us look at the integral that we encountered above
for the two-point function. After Wick rotation, we are interested in
\begin{equation}
  \label{eq:FirstExDimReg}
  I_D = \int \frac{d^Dq_E}{(2\pi)^D}\, \ESProp{q_E}{2}\, ,
\end{equation}
which is easily computed in spherical coordinates:
\begin{align}
  I_D &= \int \frac{d\Omega_D}{(2\pi)^D}\,
        \frac12 \int_0^\infty dq_E^2\, (q_E^2)^{D/2-1} \ESProp{q_E}{2}
  \\
      &= \frac{1}{(2\pi)^{D}} \frac{2\pi^{D/2}}{\Gamma(D/2)}\,
        \frac12 \left(\frac{1}{M^2}\right)^{2-D/2} \int_0^1 d\xi \,
        \xi^{1-D/2} (1-\xi)^{D/2-1} \\
  \label{eq:IDResult}
      &= \frac{1}{(4\pi)^{D/2}} \frac{\Gamma(2-D/2)}{\Gamma(2)}
        \left(\frac{1}{M^2}\right)^{2-D/2}\, .
\end{align}

\paragraph{Mathematical aside}

In deriving the above result we have made use of a number of useful
properties/tricks, which we summarise here.

\begin{enumerate}
\item The solid angle in $D$ dimension is
  \begin{equation}
    \label{eq:SoliAngleDDim}
    \Omega_D = \int d\Omega_D =
    \frac{2\pi^{D/2}}{\Gamma(D/2)}\, .
  \end{equation}
\item We made a change of integration variable:
  \begin{equation}
    \label{eq:IntegVarChange}
    \xi = \frac{M^2}{q_E^2+M^2}\, .
  \end{equation}
\item The Euler gamma function is defined as
  \begin{equation}
    \label{eq:GammaFunDef}
    \Gamma(z) = \int_0^\infty dt\, t^{z-1} e^{-t}\, .
  \end{equation}
\item It can be readily shown that
  \begin{equation}
    \label{eq:GammRec}
    \Gamma(z+1) = z \Gamma(z)\, .
  \end{equation}
\item The beta function (Euler integral of the first kind) is defined
  as
  \begin{equation}
    \label{eq:BetaFunDef}
    B(\alpha,\beta) = \frac{\Gamma(\alpha)
      \Gamma(\beta)}{\Gamma(\alpha+\beta)} =
    \int_0^1 d\xi\, \xi^{\alpha-1} \left(1-\xi\right)^{\beta-1}\, .
  \end{equation}
\item For $n\geq 0$, and small $\epsilon$, we have
  \begin{align}
    \Gamma(n+1) &= n! \, , \\
    \Gamma\left(n+\frac12\right) &= \frac{(2n)!}{n!2^{(2n)}}
                                   \sqrt{\pi}\, , \\
    \Gamma(-n+\epsilon) &= \frac{(-)^n}{n!} \left[
                          \frac{1}{\epsilon} - \gamma + \sum_{k=1}^n
                          \frac{1}{k} + O(\epsilon)
                          \right]\, ,
  \end{align}
  where $\gamma=0.5772...$ is the Euler-Mascheroni constant. 
\end{enumerate}

\paragraph{Divergences of $I_D$}

Eq.~(\ref{eq:IDResult}) provides an expression for $I_D$ which can be
extended by analytical continuation to arbitrary values of $D$. It is
interesting to note that the divergences that we identified in $D=4$
and $D=6$ appear in the regularized version as poles of the gamma
function for negative integer values of its argument. 

\paragraph{General formula}

It is useful to generalise the result in Eq.~(\ref{eq:IDResult}):
\begin{equation}
  \boxed{
  \int \frac{d^Dq_E}{(2\pi)^D}\,
  \frac{\left(q_E^2\right)^a}{\left(q_E^2+M^2\right)^b}
  = \frac{\Gamma(b-a-D/2) \Gamma(a+D/2)}{(4\pi)^{D/2} \Gamma(b)
  \Gamma(D/2)} \left(M^2\right)^{-(b-a-D/2)}\, .}
\end{equation}
Manipulations similar to the ones above allow you to derive the
general formula. 

\subsection{Structure of divergences}
\label{sec:struct-diverg}

Let us now consider the case $D=6$, and work out in more detail the
structure of the divergences that appear in $\Pi(p^2)$. As discussed
above, for $D=6$ the integral $I_D$ is logarithmically divergent, and
therefore it is convergent as soon as $D<6$. We will therefore use
dimensional regularization, and define the integral in
$D=6-2\epsilon$.

Before we start manipulating the integral, we need to do some
dimensional analysis first. For $D=6$, the scalar field has mass
dimensions
\[
  \left[\phi\right] = \frac{D-2}{2}= 2\, ,
\]
and therefore the coupling constant $g$ is dimensionless. This is a
useful property that we want to preserve; as we continue our
expressions to $D=6-2\epsilon$ we replace $g$ in the action by
$g\tilde{\mu}^\epsilon$, where $\tilde{\mu}$ is an arbitrary scale,
and $g$ remains a dimensionless coupling. This seemingly harmful
rescaling has deep consequences: the regularization procedure -- in
this case changing the number of space-time dimensions -- has
automatically introduced a new scale in the problem.

With these choices for the regulator, we find
\begin{align}
  \Pi(p^2) = \frac12 \alpha \Gamma(\epsilon-1) \int_0^1dx\,
  M^2(x,p^2)
  \left(\frac{4\pi\tilde{\mu}^2}{M^2(x,p^2)}\right)^\epsilon\, .
\end{align}

\paragraph{$\epsilon$ dependence}

We can now compute explicitly the dependence on $\epsilon$:
\begin{align}
  \Gamma(\epsilon-1) &= -\left[
                       \frac{1}{\epsilon} - \gamma + 1 +O(\epsilon)
                       \right] \, , \\
  \left(\frac{4\pi\tilde{\mu}^2}{M^2(x,p^2)}\right)^\epsilon
                     &= 1 + \epsilon \log
                       \left(\frac{4\pi\tilde{\mu}^2}{M^2(x,p^2)}\right)
                       + O(\epsilon^2)\, ,
\end{align}
and hence
\begin{align}
  \Gamma(\epsilon-1)
  \left(\frac{4\pi\tilde{\mu}^2}{M^2(x,p^2)}\right)^\epsilon
  &= - \left[
    \frac{1}{\epsilon} - \gamma +1 + \log
    \left(\frac{4\pi\tilde{\mu}^2}{M^2(x,p^2)}\right) 
    + O(\epsilon)
    \right]\, .
\end{align}
Collecting all contributions yields
\begin{align}
  \Pi(p^2) &= -\frac{\alpha}{2} \int_0^1 dx\,
             M^2(x,p^2) \left[
             \frac{1}{\epsilon} +1 + \log
             \left(\frac{4\pi\tilde{\mu}^2}{e^\gamma M^2(x,p^2)}\right) 
    + O(\epsilon)
             \right] \\
  \label{eq:ExplicitDivs}
           &= \frac{\alpha}{2} \left[
             \left(\frac{1}{\epsilon}+1\right) \left(\frac{1}{6}
             p^2-m^2\right)
             - \int_0^1dx\, M^2(x,p^2) \log\left(\frac{\mu^2}{M^2}\right)
             \right]
             +O(\epsilon) \, ,
\end{align}
where
\begin{equation}
  \label{eq:DefAlpha}
  \alpha = \frac{g^2}{(4\pi)^3}\, , \quad \mu^2 =
  \frac{4\pi}{e^\gamma} \tilde{\mu}^2\, .
\end{equation}
Eq.~(\ref{eq:ExplicitDivs}) shows explicitly the structure of the
divergences in the loop integral. They are given by the two terms
proportional to $1/\epsilon$, and are proportional to $p^2$ and
$m^2$. Understanding the structure of the divergent terms is the first
step to be able to understand how to treat them. For the time being,
we note that the divergent terms look like the contribution to the
propagator that one would obtain from interaction vertices in the
lagrangian that contain only two fields, \ie vertices like
$\partial_\mu \phi \partial^\mu\phi$ and $\phi^2$. These vertices are
usually called {\em counter terms}.

\paragraph{A geometric series}

Having evaluated $\Pi(p^2)$, we can easily re-sum an entire class of
contributions:
\begin{align}
  \label{eq:PiGeometricSeries}
  \frac{1}{i} \tDelta_F(p^2)
  =&\, 
     \begin{tikzpicture}[baseline=\plusheight]
       \begin{feynman}
         \vertex (a);
         \vertex [right= 1.0cm of a] (b);
         \diagram [horizontal=a to b, layered layout] {
           (a) --(b)
         };
       \end{feynman}
     \end{tikzpicture}
     +
     \begin{tikzpicture}[baseline=\plusheight]
       \begin{feynman}
         \vertex (a);
         \vertex [left=0.5cm of a] (i1);
         \vertex [right= 1.0cm of a] (b);
         \vertex [right=0.5cm of b] (f1);
         \diagram [horizontal=a to b, layered layout] {
           (i1) -- (a)
           -- [half left] (b) 
           -- [half left] (a) ,
           (f1) --(b)
         };
       \end{feynman}
     \end{tikzpicture}
     +
     \begin{tikzpicture}[baseline=\plusheight]
       \begin{feynman}
         \vertex (a);
         \vertex [left=0.5cm of a] (i1);
         \vertex [right= 1.0cm of a] (b);
         \vertex [right=0.5cm of b] (f1);
         \vertex [right= 1.0cm of f1] (b1);
         \vertex [right=0.5cm of b1] (f2);
         \diagram [horizontal=a to b, layered layout] {
           (i1) -- (a)
           -- [half left] (b) 
           -- [half left] (a) ,
           (b) --(f1)
           -- [half left] (b1)
           -- [half left] (f1),
           (b1) -- (f2);
         };
       \end{feynman}
     \end{tikzpicture}
     + \ldots \\
  =& \frac{1}{i} \tDelta(p^2) +
     \frac{1}{i} \tDelta(p^2) \left[\left(i \Pi(p^2)\right)
     \left(\frac{1}{i} \tDelta(p^2) \right) \right] + \nonumber \\
  & \quad\quad + 
     \frac{1}{i} \tDelta(p^2) \left[\left(i \Pi(p^2)\right)
     \left(\frac{1}{i} \tDelta(p^2) \right) \right]^2 +
    \ldots \\
  =& \frac{1}{i} \tDelta(p^2) \; \sum_{k=0}^\infty \left[\Pi(p^2)
     \tDelta(p^2)\right]^k \\ 
  =&  \frac{1}{i} \tDelta(p^2)\; \frac{1}{1-\Pi(p^2) \tDelta(p^2)}\, .
\end{align}
Hence the net result of the sum yields
\begin{equation}
  \label{eq:FullPropOneLoop}
  \tDelta_F(p^2) = \frac{1}{p^2-m^2-\Pi(p^2) + i\epsilon}\, ,
\end{equation}
where we have reintroduced the $i\epsilon $ term in the denominator
(not to be confused with the parameter $\epsilon $ of DimReg).

\paragraph{Comparison with K-L}

According to the Källén-Lehmann representation of the propagator, we
expect to find a pole for $p^2=\mphys^2$, with the corresponding
residue being equal to one. Two observations are in order.
\begin{enumerate}
\item 
  In order to have such a pole we need
  \begin{equation}
    \label{eq:MphysPole}
    \mphys^2 - m^2 - \Pi(\mphys^2) = 0
  \end{equation}
  to hold, which clearly shows that $\mphys^2\neq m^2$.
\item
  Similarly we can
  compute the residue of the propagator at $p^2=\mphys^2$ by expanding
  the denominator in $p^2$ around $\mphys^2$,
  \begin{equation}
    \label{eq:ResDenExp}
    p^2-m^2-\Pi(p^2) =
    \left(p^2 - \mphys^2\right) \left[1 - \Pi'(\mphys^2)\right] +
    O\left(\left(p^2 - \mphys^2\right)^2\right)\, ,
  \end{equation}
  which in turn yields
  \begin{align}
    \mathrm{Res}_{\mphys^2} \tDelta(p^2)
    &= \lim_{p^2\to\mphys^2} \left(p^2 - \mphys^2\right) \tDelta(p^2) \\
    &= \frac{1}{1 - \Pi'(\mphys^2)}\, .
  \end{align}
  The latter equation shows that the field $\phi$ that appears in the
  Lagrangian does not have the normalization required to agree with the
  K-L representation.
\end{enumerate}
These two observations suggest a general line of thinking: the fields
and couplings that appear in the Lagrangian, the so-called {\em bare
  fields} and {\em bare coupings} are not physical. The physical
variables need to be defined according to some well-specified
prescription. The process of specifying these quantities is known as
{\em renormalization} of the theory.

\section{Renormalization}
\label{sec:renormalization}

\subsection{Renormalized perturbation theory}
\label{sec:renorm-pert-theory}

\paragraph{Renormalization of the field}

As suggested by the comparison with K-L, we define a {\em
  renormalized field}
\begin{equation}
  \label{eq:RenormField}
  \phi(x) = Z^{1/2} \phi_R(x)\, ,
\end{equation}
where $Z$ is the renormalization constant of the field.  In terms of
the new field the Lagrangian is
\begin{align}
  \mathcal{L}= \frac12 \partial_\mu \phi_R \partial^\mu \phi_R -
  \frac12 m^2 Z \phi_R^2 + \frac{g}{3!} Z^{3/2} \phi_R^3 +
  \frac12 \delta_Z \partial_\mu \phi_R \partial^\mu \phi_R\, ,
\end{align}
where we have introduced $\delta_Z=Z-1$.

\paragraph{Renormalization of mass and coupling}

We can also introduce a renormalized mass and a renormalized coupling
\begin{equation}
  \label{eq:RenormMass}
  m^2 Z = Z_m m_R^2\, ,\quad g Z^{3/2}= Z_g g_R
\end{equation}
where we have introduced two new renormalization constants $Z_m$ and
$Z_g$.

\paragraph{Renormalized perturbation theory}

We can rewrite the Lagrangian one more time as
\begin{align}
  \mathcal{L}= \frac12 \partial_\mu \phi_R \partial^\mu \phi_R -
  \frac12 m_R^2 \phi_R^2 + \frac{Z_g g_R}{3!} \phi_R^3 +
  \frac12 \delta_Z \partial_\mu \phi_R \partial^\mu \phi_R
  - \frac12 \delta_m \phi_R^2
  \, ,
\end{align}
where $\delta_m=Z_m-1$. The expression above for the Lagrangian looks
identical to the bare one, except that fields and couplings are now
renormalized, and there two counter terms proportional to $\delta_Z$
and $\delta_m$ respectively. We can therefore define the path integral
in terms of renormalized quantities; this will lead to 
{\em renormalized perturbation theory}. 

We can now compute $\Pi(p^2)$ in renormalized perturbation theory, the
result is identical to the one obtained before, plus the contribution
of the counter terms:
\begin{equation}
  \label{eq:PiRenormPertTh}
  \Pi(p^2)=\frac{\alpha}{2}\left[
    \left(\frac{1}{\epsilon}+1\right)
    \left(\frac{1}{6}p^2-m_R^2\right) +
    \int_0^1dx\, M^2\log\left(\frac{M^2}{\mu^2}\right)
  \right] + \delta_Z p^2 - \delta_m m_R^2 + O(\alpha^2)\, .
\end{equation}
Is it useful to rewrite the equation above as:
\begin{align}
  \Pi(p^2)
  =& \frac{\alpha}{2}\int_0^1dx\,
     M^2\log\left(\frac{M^2}{m_R^2}\right) +
     \left[
     \frac{\alpha}{6} \left(\frac{1}{2\epsilon} + \log(\mu/m_R) +
     \frac12 \right) + \delta_Z
     \right] p^2 - \nonumber \\
   & - \left[
     \alpha\left(\frac{1}{2\epsilon} + \log(\mu/m_R) +
     \frac12 \right) + \delta_m
     \right] m_R^2 + O\left(\alpha^2\right)\, .
\end{align}
Note that at this stage we have not yet defined the renormalization
constants, and therefore $\delta_Z$ and $\delta_m$ are free parameters
that need to be fixed.  Note also that by choosing
\begin{align}
  \delta_z =& -\frac{\alpha}{6} \left(\frac{1}{2\epsilon} + \log(\mu/m_R) +
              \frac12 + \kappa_Z \right) + O\left(\alpha^2\right)\, , \\
  \delta_m =& -\alpha\left(\frac{1}{2\epsilon} + \log(\mu/m_R) +
     \frac12 + \kappa_m\right) + O\left(\alpha^2\right)\, , 
\end{align}
where $\kappa_Z$ and $\kappa_m$ are finite constants, we obtain a
value for $\Pi(p^2)$ which is finite when $\epsilon\to 0$, and
independent of the arbitrary scale $\mu$.

\paragraph{Renormalization conditions}

In order to fully determine the renormalization constants, we need to
specify a so-called {\em renormalization scheme}. The renormalization
scheme is defined by imposing a number of conditions that are
sufficient to determine all the renormalization constants. Clearly these
conditions are necessary in order to have a predictive framework.

\subsection{Renormalization scheme}
\label{sec:renorm-scheme}

