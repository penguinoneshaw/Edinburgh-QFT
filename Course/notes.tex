\documentclass[bibliography=totoc]{notes}

\addto\captionsUKenglish{\renewcommand{\chaptername}{Lecture}}
\KOMAoptions{
  titlepage=firstiscover
}
\addbibresource{QFT.bib}

\title{Quantum Field Theory}
\coursecode{PHYS11065}
\author{L.~Del Debbio}
\version{0.2}
\school{
  School of Physics and Astronomy
  }
  \date{November 2018}
  
  \makeindex

\begin{document}
%\makeatletter
\begin{titlepage}
\ifdefined\@school
        \begin{minipage}{0.6\textwidth}
        \begin{flushleft}
                {\usekomafont{disposition} \Huge  \linespread{1.15} \@school \par}
        \end{flushleft}
        \end{minipage}\hfill
        \begin{minipage}{0.3\textwidth}
                \includegraphics[width=40mm,height=40mm]{Includes/crest.eps}
        \end{minipage}
\else
{\centering \includegraphics[width=40mm,height=40mm]{Includes/crest.eps}\\}
\fi
\vspace*{4cm}

{\centering
\ifdefined\@coursecode
{\LARGE\rmfamily\scshape \@coursecode\\}
\fi
{\Huge\usekomafont{title} \@title\\}
\vspace{\fill}
{\Large \@author \\[1ex] \LARGE\bfseries \@date \ifdefined\@version \\[2ex] \large Version \@version\fi\\}
}
\end{titlepage}
\makeatother

\maketitle
\pagenumbering{roman}
\clearpage
\tableofcontents

\vspace{\fill}

This document and its source code are licensed under a \href{https://www.gnu.org/licenses/gpl-3.0.en.html}{GNU GPLv3} License.

% The maker of Tikz-Feynman asks that you reference his paper when you use it.
The Feynman diagrams are generated using \texttt{tikz-feynman}\supercite{ELLIS2017103}.

\nocite{*}
\printbibliography[keyword=recommended,title={Recommended Textbooks},omitnumbers=true]
\pagenumbering{arabic}

\subfile{intro}
\subfile{lecture0}
\subfile{lecture1}
\subfile{lecture2}
%\input{Sections/ScalarPI}

\subfile{lecture3}
\subfile{lecture4}
\subfile{lecture5}
\subfile{lecture6}

\chapter{Divergences II --- the Vertex Function}
\label{cha:diverg-ii-vert}


\chapter{Renormalization}
\label{cha:renormalization}
% \input{Sections/Renorm.tex}


\chapter{Renormalization Group}
\label{cha:renorm-group}

\chapter{Scales - Decoupling}
\label{cha:scales-decoupling}

\printindex
\printbibliography[notkeyword=recommended]

\end{document}

%%% Local Variables:
%%% mode: latex
%%% TeX-master: t
%%% End:
