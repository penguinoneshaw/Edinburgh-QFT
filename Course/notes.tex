\documentclass[bibliography=totoc,titlepage=firstiscover]{scrreprt}


\pagestyle{headings}

\usepackage{mathpazo}

\usepackage{amsmath}
\usepackage{amsfonts}
\usepackage{amssymb}
\usepackage{dsfont}
\usepackage{pifont}
%\usepackage{bbold}
\usepackage{graphicx}
\usepackage{epstopdf}
\usepackage{epsfig}
%\usepackage{bibunits}
%\usepackage{theorem}
\usepackage[framed]{ntheorem}
\usepackage{framed}
%\usepackage{showlabels}
\usepackage{makeidx}
\usepackage{simplewick}
\usepackage[compat=1.1.0]{tikz-feynman}
\usepackage{slashed}
\usepackage{appendix}
\usepackage[hidelinks]{hyperref}
\usepackage[arrowdel]{physics}
\usepackage{setspace}
\usepackage[sorting=ynt,bibencoding=utf8,giveninits,uniquename=init,backend=biber,defernumbers=true]{biblatex}

\tikzfeynmanset{warn luatex=false}

\newcommand{\tick}{\ding{52}}
\newcommand{\notick}{\ding{56}}
\newcommand{\D}{\displaystyle}

\def\bfx{{\mathbf x}}
\def\bfxp{{\mathbf x^\prime}}
\def\bfy{{\mathbf y}}
\def\bfyp{{\mathbf y^\prime}}
\def\bfp{{\mathbf p}}
\def\bfpp{{\mathbf p^\prime}}
\def\ddt{\frac{\mathrm{d}}{\mathrm{d}t}}
\def\ddtt{\frac{d^2}{dt^2}}
\def\ie{\emph{ i.e.}\ }
\def\eg{\emph{ e.g.}\ }
\def\viz{\emph{ viz.}\ }
\def\matF{\mathcal F}
\def\matE{\mathcal E}
\def\GL{\mathrm{GL}}
\def\kpsi{|\psi\rangle}
\def\kpsione{|\psi_1\rangle}
\def\kpsitwo{|\psi_2\rangle}
\def\kpsionep{|\psi_1^\prime\rangle}
\def\kpsitwop{|\psi_2^\prime\rangle}
\def\kpsii{|\psi_i\rangle}
\def\kpsin{|\psi_n\rangle}
\def\kpsip{|\psi^\prime\rangle}
\def\bpsi{\langle\psi |}
\def\bpsione{\langle\psi_1 |}
\def\bpsitwo{\langle\psi_2 |}
\def\bpsii{\langle\psi_i |}
\def\bpsip{\langle\psi^\prime |}
\def\kphi{|\phi\rangle}
\def\kphione{|\phi_1\rangle}
\def\kphitwo{|\phi_2\rangle}
\def\kphii{|\phi_i\rangle}
\def\kphip{|\phi^\prime\rangle}
\def\bphi{\langle\phi |}
\def\bphione{\langle\phi_1 |}
\def\bphitwo{\langle\phi_2 |}
\def\bphii{\langle\phi_i |}
\def\bphip{\langle\phi^\prime |}
\def\bchi{\langle\chi |}
\def\bchione{\langle\chi_1 |}
\def\bchitwo{\langle\chi_2 |}
\def\bchii{\langle\chi_i |}
\def\bchip{\langle\chi^\prime |}
\def\kjm{|j,m\rangle}
\def\tr{\mathrm{Tr}}
\def\id{\mathds{1}}

{\theoremstyle{plain} \theorembodyfont{\rmfamily} \newframedtheorem{Ex}{Exercise}[section]}
{\theoremstyle{plain} \theorembodyfont{\rmfamily} \newtheorem{Def}{Definition}[section]}
{\theoremstyle{plain} \theorembodyfont{\rmfamily} \newtheorem{Thm}{Theorem}[section]}

\newcommand{\clearemptydoublepage}{\newpage{\pagestyle{empty}\cleardoublepage}}
\newcommand{\HRule}{\rule{\linewidth}{0.5mm}}
\newcommand{\iu}{\underline{i}}
\newcommand{\ju}{\underline{j}}
\newcommand{\ku}{\underline{k}}
\newcommand{\ru}{\underline{r}}
\newcommand{\pu}{\underline{p}}
\newcommand{\Lu}{\underline{L}}
\newcommand{\Ju}{\underline{J}}
\newcommand{\lap}{\nabla^2}
\newcommand{\ad}{\hat{a}}
\newcommand{\ac}{\hat{a}^\dagger}
\newcommand{\re}{\mathrm{Re}}
\newcommand{\pref}[1]{(\ref{#1})}
\newcommand{\Eqref}[1]{Eq.~(\ref{#1})}
\newcommand{\del}{\v{\nabla}}				% Underlined del
\def\plusheight{-\the\dimexpr\fontdimen22\textfont2\relax}
\newcommand{\mphys}{m_\mathrm{phys}}
\newcommand{\psibar}{\bar{\psi}}
\newcommand{\etabar}{\bar{\eta}}
\newcommand{\munu}{{\mu\nu}}
\newcommand{\tDelta}{\tilde{\Delta}}
\newcommand{\SProp}[1]{\frac{1}{#1^2-m^2+i\epsilon}}
\newcommand{\ESProp}[2]{\frac{1}{\left(#1^2+m^2\right)^#2}}
\newcommand{\tphi}{\tilde{\phi}}
\newcommand{\tj}{\tilde{J}}
\newcommand{\tchi}{\tilde{\chi}}

\KOMAoption{chapterprefix}{true}

\KOMAoptions{
paper=a4,
fontsize=11pt,
parskip=half-,
BCOR=.5cm,
pagesize=auto,
pagesize=pdftex,
headinclude=false,
footinclude=false,
DIV=12
}

\usepackage{subfiles}
\setkomafont{disposition}{\rmfamily\scshape\bfseries}

\makeatletter
\newcommand{\school}[1]{\def\@school{#1}}
\newcommand{\version}[1]{\def\@version{#1}}
\newcommand{\coursecode}[1]{\def\@coursecode{#1}}
\makeatother

\addbibresource{QFT.bib}


\makeindex
\title{Quantum Field Theory}
\coursecode{PHYS11065}
\author{L.~Del Debbio}
\version{0.2}
\school{
  School of Physics and Astronomy
}
\date{November 2018}

\begin{document}
\makeatletter
\begin{titlepage}
\ifdefined\@school
        \begin{minipage}{0.6\textwidth}
        \begin{flushleft}
                {\usekomafont{disposition} \Huge  \linespread{1.15} \@school \par}
        \end{flushleft}
        \end{minipage}\hfill
        \begin{minipage}{0.3\textwidth}
                \includegraphics[width=40mm,height=40mm]{Includes/crest.eps}
        \end{minipage}
\else
{\centering \includegraphics[width=40mm,height=40mm]{Includes/crest.eps}\\}
\fi
\vspace*{4cm}

{\centering
\ifdefined\@coursecode
{\LARGE\rmfamily\scshape \@coursecode\\}
\fi
{\Huge\usekomafont{title} \@title\\}
\vspace{\fill}
{\Large \@author \\[1ex] \LARGE\bfseries \@date \ifdefined\@version \\[2ex] \large Version \@version\fi\\}
}
\end{titlepage}
\makeatother

\pagenumbering{roman}
\tableofcontents

\vspace{\fill}

This document and its source code are licensed under a \href{https://www.gnu.org/licenses/gpl-3.0.en.html}{GNU GPLv3} License.

% The maker of Tikz-Feynman asks that you reference his paper when you use it.
The Feynman diagrams are generated using \texttt{tikz-feynman}\supercite{ELLIS2017103}.

\nocite{*}
\printbibliography[keyword=recommended,title={Recommended Textbooks},omitnumbers=true]
\pagenumbering{arabic}

\subfile{intro}
\subfile{lecture0}
\subfile{lecture1}
\subfile{lecture2}
%\input{Sections/ScalarPI}

\subfile{lecture3}
\subfile{lecture4}
\subfile{lecture5}
\subfile{lecture6}


\chapter{Divergences II --- the Vertex Function}
\label{cha:diverg-ii-vert}


\chapter{Renormalization}
\label{cha:renormalization}
% \input{Sections/Renorm.tex}


\chapter{Renormalization Group}
\label{cha:renorm-group}

\chapter{Scales - Decoupling}
\label{cha:scales-decoupling}

\printindex
\printbibliography[notkeyword=recommended]

\end{document}

%%% Local Variables:
%%% mode: latex
%%% TeX-master: t
%%% End:
